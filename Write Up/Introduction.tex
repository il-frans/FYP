\chapter{Introduction}

% Intro to the areas
The gamification of areas of activity such as marketing, problem solving and education \cite{michael2005serious} has validated the use of serious games beyond their initial military use in training strategic skills \cite{djaouti2011classifying}.  Serious games simulate real-world processes designed for the purpose of solving a problem, making their main purpose that of training or educating users. Their popularity has been steadily increasing, as has their adoption, with military \cite{djaouti2011classifying} and emergency service providers (e.g. firefighters \cite{michael2005serious}) employing them to train for specific scenarios that might be encountered on the respective jobs. Motorsports cover a broad range of activities and vehicles, and as with all major forms of sporting activities, require training and dedication, with a pedagogic aspect arising in rote learning and mentoring by experts. The arenas in which motorsport events take place are called circuits; there is a large selection of the latter, ranging from purposely-built race tracks to public roads to natural formations such as hills and quarries. There is also a diverse selection of vehicles that take part in motorsports, with the greatest demarcation existing between motorbikes and cars. The focus of this dissertation is that of unifying serious games and motorsport racing; specifically, it will try to show whether a serious game is a powerful enough pedagogical tool that can be used to tangibly improve the performance of race drivers. The scope of the project is limited to four-wheeled cars racing on purposely-built confined circuits with a smooth tarmac surface.  

% How the project will be tackled
%This study will take an exploratory and descriptive research approach by first looking into the equipment and environment race drivers work with and their properties, followed by identifying what skills race drivers must master, why these skills are important and the impact they have on their performance. Sim racing hardware will be defined and hardware from various price ranges will be explored. Academic literature will be consulted particularly on what makes a serious game, their applications and state of the art within the domain of motorsport and driving in general. Possible training mechanism will be investigated, aiming to develop an artefact which given telemetry data, it is able to identify erroneous driving in the context of racing and manage to convey to the user what should be done in order to avoid such behaviour.  Evaluation strategies will be investigated and one is to be implemented in order to be able to quantify the effectiveness of the artefact. Finally, findings from the evaluation strategy will be presented and any claims derived from the findings will be put forward.

\section{Aims and Objectives}
The aim of this dissertation is to demonstrate whether serious games can be used to teach a particular brand of motorsports. In order to establish whether this is possible or otherwise, the following objectives have been set out:

\begin{enumerate}
	\item Research and develop a model for assessing race driver performance (see chapters \ref{chp:background}, \ref{chp:literature-review})
	\item Develop a system of heuristics for providing feedback from the aforementioned model (see Chapter \ref{chp:design-implementation})
	\item Devise an experimental methodology for evaluating how effective the feedback model is (if at all) in training race drivers (see Chapter \ref{chp:methodology})
	\item Design and develop the main software and support tools for integrating the feedback model into a racing simulation game (see \ref{chp:design-implementation})
	\item Carry out a user study based on devised experimental methodology to gather data for analysis (see Chapter \ref{chp:evaluation})
	\item Perform statistical analysis on the gathered data to answer research question (see Chapter \ref{chp:evaluation})
\end{enumerate}

%This project sets out to answer whether a context-based feedback system can improve the skills of a participant while also being cost effective This is to be achieved by means of evaluation process and statistical analysis. The objectives are broken down as
%
%1. Assess the magnitude of the improvement when a context-based feedback system is introduced.
%
%2. Analyse any statistical significance after using a context-based feedback system.

\section{Dissertation Overview}
This dissertation is structured as follows:
\begin{description}
	\item [Chapter \ref{chp:background}] provides a brief overview into motorsports, together with an exposition of important concepts such as the racing line, cornering and braking, telemetry and simulation racing rigs is then given.
	\item [Chapter \ref{chp:literature-review}] reviews literature in the area of serious games, with emphasis on their pedagogical aspect. This is followed by an overview of simulation racing, focusing on the genre's intersection between entertainment and pedagogical factors.
	\item [Chapter \ref{chp:methodology}] provides a detailed exposition of the methodology used throughout this work. 
	\item [Chapter \ref{chp:design-implementation}] gives an in-depth description of the design and implementation of the software artefact employed in this work, starting with an overview of the core requirements for the main software artefact which we refer to as Telemetry Assisted Racing (TeAR), before presenting the system architecture and delving into each module. 
	\item [Chapter \ref{chp:evaluation}] presents the results of the user study and questionnaires, and closes with a discussion on these results.
	\item [Chapter \ref{chp:conclusion}] concludes this dissertation, summarising the findings and proposing avenues for future work.
\end{description}