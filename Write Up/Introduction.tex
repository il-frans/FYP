\chapter{Introduction}

% Intro to the areas
The gamification of areas of activity such as marketing, problem solving and education \cite{michael2005serious} has validated the use of serious games beyond their initial military use in training strategic skills \cite{djaouti2011classifying}.  Serious games simulate real-world processes designed for the purpose of solving a problem, making their main purpose that of training or educating users. Their popularity has been steadily increasing, as has their adoption, with military \cite{djaouti2011classifying} and emergency service providers (e.g. firefighters \cite{michael2005serious}) employing them to train for specific scenarios that might be encountered on the respective jobs. Motorsports cover a broad range of activities and vehicles, and as with all major forms of sporting activities, require training and dedication, with a pedagogic aspect arising in rote learning and mentoring by experts. The arenas in which motorsport events take places are called circuits; there is a large selection of the latter, ranging from purposely built race tracks to public roads to natural formations such as hills and quarries. There is also a diverse selection of vehicles that take part in motorsports, with the greatest demarcation existing between motorbikes and cars. The focus of this dissertation is that of unifying serious games and motorsport racing; specifically, it will try to show whether a serious game is a powerful enough pedagogical tool that can be used to tangibly improve the performance of race drivers. The scope of the project is limited to four-wheeled cars racing on purposely-built confined circuits with a smooth tarmac surface.  

% How the project will be tackled
This study will take an exploratory and descriptive research approach by first looking into the equipment and environment race drivers work with and their properties, followed by identifying what skills race drivers must master, why these skills are important and the impact they have on their performance. Sim racing hardware will be defined and hardware from various price ranges will be explored. Academic literature will be consulted particularly on what makes a serious game, their applications and state of the art within the domain of motorsport and driving in general. Possible training mechanism will be investigated, aiming to develop an artefact which given telemetry data, it is able to identify erroneous driving in the context of racing and manage to convey to the user what should be done in order to avoid such behaviour.  Evaluation strategies will be investigated and one is to be implemented in order to be able to quantify the effectiveness of the artefact. Finally, findings from the evaluation strategy will be presented and any claims derived from the findings will be put forward.

\section{Aims and Objectives}
\section{Dissertation Overview}