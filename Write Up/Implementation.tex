\section{Implementation}

\subsection{Supporting tools}

\subsection{Track Splicer}
This project is designed to work with any race track given specific track meta data is provided to aid the feedback system processing. The meta data is split into two files, ‘raceline.csv’ and ‘sections.csv’. ‘raceline.csv’ is a coma separated file containing a list of records denoting coordinates on track which make up the race. For this particular study the race line files have been generated from an .ai file which is supplied with Assetto Corsa. Each track has an associated 'ideal\_line.ai' file associated with it. The ai file contains raw bytes, which through manual investigation of the file in hex view, it was noted the file is made of a header part of 36 bytes, followed by a sequence of repeating records of 20 bytes each. These records contain four floats and one 32bit integer, storing the data which is required in the ‘raceline.csv’ file. The records in the ai file are read via a custom developed command line tool and translated into the csv format required by the feedback system. 	

\begin{center}
    \begin{tabular}{ | l | l | p{5cm} |}
    \hline
    Field Name & Description \\ \hline
    ID & Ordered unique ID \\ \hline
    Distance & The distance from the start of the lap \\ \hline
    X & X coordinates \\ \hline
    Y & Y coordinates \\ \hline
    Z & Z coordinates \\ \hline
    \end{tabular}
\end{center}

Moving on to the ‘sections.csv’, this file is also a coma separated file containing a sequence of records. These records denote corners and straights which make up the track, and will be used to compute any feedback which is specific to straights or corners. In order to generate this file a tool has been developed which loads the ‘raceline.csv’ and computes the rate of change from one data point to the next. Depending on the rate of change the points are classified as either part of a straight section or as a corner section. This is done by taking three points, p1, p2 and p3 from which two vectors are generated v1 and v2. V1 is the vector from p1 to p2, and v2 is the vector from p2 to p3. Then, v1 and v2 are normalised and the dot product computed which give out the rate of change in radians. The pseudo code for this is shown below.\\

for (i = 0; i < racelinePoints.Count - 2; i++)\\
\{\\
Vector2 v1 = racelinePoints[i].GetVectorToPoint(racelinePoints[i+1]);\\
Vector2 v2 = racelinePoints[i+1].GetVectorToPoint(racelinePoints[i+2]);\\
V1 = Vector2.Normalize(v1);\\
V2 = Vector2.Normalize(v2);\\
float dotProduct = Vector2.Dot(v1, v2);\\
double difference = Math.Acos(dotProduct);\\
\}\\

The corner mid-point can be defined as the highest section of the arch. In order to find this, it is simply a matter of finding the highest possible vector dot product from the section starting point, to the end of the section. The pseudo code is shown below.\\

Point p = endPoint - startPoint;\\
Vector2 n = new Vector2(-p.Y, p.X);\\
Int idOfMax = -1;\\
float max = -1;\\
for (i = trackSection.StartPoint; i <= trackSection.EndPoint; i++)\\
\{\\
p = \_RacingLine[i] - startPoint;\\
float result = Vector2.Dot(new Vector2(point.X, point.Y), n);\\
result = Math.Abs(result);\\
if (result > max)\\
\\{\\
max = result;
idOfMax = i;
\\}\\
\}

\ref{fig:TrackSplicerTool} shows the tool in action which also provides a visual representation of the race line. Corner sections are shows are show are red dots, with green dots used to highlight a corner’s mid-point and straights are shown in blue.

\begin{figure}[!htb]
	\centering
	\includegraphics[height=7cm]{images/tracksplicertool}
	\caption{Track splicer tool}
	\label{fig:TrackSplicerTool}
\end{figure}

\subsection{Spatial Querying}
As previously mentioned it is important for the feedback system to be able to carry out fast spatial querying operations. A query for the nearest race line data point based relative to the current position of the car is required to be carried out multiple times per second. Thanks to the implementation of a quad tree, the search guaranteed to take place in O(logn), while insertion is done O(nlogn) however, this is not too relevant as all insertion are carried out before the feedback system starts its computations. This structure allows the feedback to quickly calculate in which section of the track the car is located, the nearest race line data point and how far from the race line the car is.

\begin{figure}[!htb]
	\centering
	\includegraphics[height=7cm]{images/QuadTree}
	\caption{Visual representation for part of the quad tree}
	\label{fig:	QuadTree}
\end{figure}

VR Compatibility issues
VR not being able to render on screen native assetto corsa apps
Having to go with audio

Debugging and Testing
	Debug Info
	Unit Tests