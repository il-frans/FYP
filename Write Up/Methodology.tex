\section{Methodology}

\subsection{Sim setup}
	The simulation setup has to mimic the real world as much as possible yet be cost effective for it to be viable for consumers to use. The driving model is to be simulated by a consumer available off the shelf sim-racing game which can provide, realistic car handling and characteristics  models and real life racing tracks which are laser scanned for maximum accuracy. The chosen sim needs also to simply an interfacing layer which developers can connect to in order to consume telemetry data. The driving controls are to be provided by means of a sim racing steering wheel, pedals and shifter which is attached to a custom built racing rig and a racing seat. The combination of this hardware will hopefully make the evaluation experience for participants an entertaining and enjoyable one, while also being able to give a genuie sense of racing round a race track. Virtual reality was initial planned to be used as it provides a good sense of immersion  however, after preliminary tests with the Oculus rift DK2 headset it was noted that testers were getting dizzy and some had eye irritation which caused virtual reality to be abounded and opt for a more traditional single screen setup.

\subsection{Feedback System}
	The feedback system will rely heavily on the quality of telemetry data it can get. This means, the data has to be, supplied in real time, ordered, accurate and carry enough information for feedback to be generated. Telemetry data is to be analysed in real time from which the system will be able to pick up on mistakes which the driver is making while going round the racing circuit. What feedback the user is given is determined by the skill level one has. During the first lap, the system is to determine the skill level of the driver by starting off checking for basic mistakes and work the way up to more advanced ones. As such the feedback will be focused to guide the driver into mastering basic skills such as the racing line and progressing to more advance techniques as the user improves.

\subsection{Evaluation}

\subsubsection{Experiment structure}
	In order to evaluate the effectiveness of the system a user study needs to take place. Participants are to be split randomly into two groups. One group will be referred as the feedback group, the other will be referred to as the base group. All participants, regardless of in which group they have been assigned, will be given one hour slots in which they are asked to race around a track. The one hour slots are to be divided in sub sessions to be carried out in the following order. Five minutes to get the rig setup and adjusted for the participant. This might require having to move the seat further back or forward in order for the participant to be more comfortable. During this time the participant will also be asked to fill in a pre-experiment survey, be given information about the racing rig usage and care, and how the one hour slot is divided. Ten minutes of driving are then carried out, the aim of this session is to allow the participant to get familiar with the rig, track and car. Once the first session of driving is carried out the participant is told in which group he/she has been assigned. The participant is also shown a picture of a typical race line through a corner and a brief explanation is given. This is done to make it easier on the feedback group to better understand the auditory race line feedback which might be given. As for the base group, the same picture and information are given out as to have both groups provided with the same information and ensuring any learning is done by means of the feedback system. Another ten minutes of driving follow, with the same setup as the previous one which gives participants the chance to keep practicing and improving. A final five minutes of driving will take place, during this session no one will have the feedback system enabled. The aim of this session is to give the participant a final chance to improve upon his/her time. in addition for the feedback group, data from the last five minutes could point out if the participant was able to learn any techniques and use them with out being given further feedback. Finally the participant is given the post survey to fill in.

\subsubsection{Surveys}
	Participants will be accepted from any background and level of relevant experience. A pre experiment survey will be designed as it is beneficial to be able to distinguish between clusters of participants who are experienced with sim racing, those who play other forms of racing games, those who don't have any experience at all and anyone who has real life experience with motorsport. By gathering this information it will be possible to identify characteristics such as whether the feedback given was more beneficial to a particular group, whether the rig setup was more problematic to use for some and if real life motorsport experience can be translated onto the sim setup. Another survey is required to gather feedback from the participants point of view regarding the experiment setup. The feedback group will be asked to fill in a extra section asking for their impressions on the feedback system. This section will mostly focusing on trying to understand if participants felt the feedback system was helpful and if there is something which can be improved upon. The participants responses will be mapped to the telemetry data from which responses can be validated. In the case participants feel the feedback is not accurate, feedback is not helpful or any other feedback can be correlated back to the telemetry data.
	
\subsubsection{Track and Car choice}
	The track and car choice is to made based on suggestion from the literature review. The car should not be too powerful as it would be harder to control. Another characteristic to take into account is whether to choose a front wheel drive or a rear wheel drive one. Rear wheel drive cars tend to be less predictable in corners, while front wheel ones tend to understeer more resulting in more predictable cornering which in turn makes front wheel drive easier to drive. The track should be as flat and smooth as possible. This makes it easier to drive the circuit as an uneven surface may upset the car making it lose control. The track also needs to have wide run off areas. these areas are located along the circuit where racers are most likely to unintentionally depart from the prescribed course. This gives the racers time to slow down before colliding with walls.
	
\subsubsection{Data Analysis}

	While the sessions are taking place, telemetry data will be logged and stored for off line analysis. The main data which are of interests to this project are lap times. Lap times can give a good indication of the participant's skill and performance. In particular the rate at which the participant's improve, if there is any improvement at all. The analysis needs to focus on clustering the data into two main groups, the base group and the feedback group, from these the best lap time of each user is to be considered for each session. The aim is to find any statistical difference in the rate of improvement between the base group and the feedback group. However as much as the ultimate goal is to improve the lap time, one must first improve on specific skills. It is possible for a participant to improve on a skill area while hindering another. In such case it would be interesting to analyse the telemetry data from which it could be determined if a participant did improve or hinder certain areas which could explain why lap times did not improve or get worst.
