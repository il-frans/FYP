\section{Methodology}

\subsection{Evaluation}
	User study

\subsubsection{Surveys}
	Participants will be accepted from any background and level of relevant experience. A pre experiment study will be designed as it is beneficial to be able to distinguish between clusters of participants who are experienced with sim racing, those who play other forms of racing games, those who don't have any experience at all and anyone who has real life experience with motorsport. By gathering this information it will be possible to identify characteristics such as whether the feedback given was more beneficial to a particular group, whether the rig setup was more problematic to use for some and if real life motorsport experience can be translated onto the sim setup. Another survey is required to gather feedback information from participants. 

		Post expiremnts

\subsubsection{Experiment structure}
	Participants are to be split randomly into two groups. One group will be referred as the feedback group, while the other will be referred to as the base group. All participants, regardless of in which group they have been assigned, will be given one hour slots in which they are asked to race around a track. The one hour slots are to be divided in sub sessions to be carried out in the following order. Five minutes to get the rig setup and adjusted for the participant. This might require having to move the seat further back or forward in order for the participant to be more comfortable. During this time the participant will also be asked to fill in the pre-experiment survey, be given information about the racing rig and how the one hour is divided. Ten minutes of driving are then carried out, the aim of this session is to allow the participant to get familiar with the rig, track and car. Once the first session of driving is carried out the participant is told in which group he/she has been assigned. The participant is also shown a picture of a typical race line through a corner and a brief explanation is given. This is done to make it easier on the feedback group to better understand the auditory race line feedback which might be given. As for the base group, the same picture and information are given out as to have both groups provided with the same information and having any learning done by means of the feedback system. Another ten minutes of driving follow, with the same setup as the previous one. Then, there is a final 5 minutes of driving, during this session no one has the feedback enabled. The aim of this session is to give the participant a final chance to improve upon his/her time. in addition for the feedback group, data from the last five minutes could point out if the participant was able to learn any techniques and use them with out being given further feedback. Finally the participant is given the post survey to fill.
		
	The track and car choice is to be the same for everyone, the choice is to be made before the start of the experiments. 
	
	
	Car choice
	Track choice