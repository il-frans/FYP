\chapter{Background}
\label{sec:background}

This chapter introduces motorsport racing and a number of related concepts that are essential in gaining an understanding of this work. The chapter opens with a brief overview of the sport, followed by an exposition of important concepts like the \emph{racing line}, cornering and braking. A discussion ensues, wherein \emph{understeer} and \emph{oversteer} and explained. A short overview of telemetry is then provided. The chapter concludes with a general introduction to simulation racing rigs.

\section{Motorsport racing}
In sports, individuals or groups compete to be the first to achieve a particular objective. In circuit motorsport racing, motorised vehicles go round a course for a set number of times. There are varies racing disciplines or series, each one having its own specific rules. However, at the core, participants in all disciplines aim to complete a full lap of the circuit in the shortest time. Some disciplines focus on achieving one fast lap, such as time trials, while others focus on achieving the least amount of time across a fixed number of laps, such as FIA's Formula 1 series. This dissertation will focus on one such discipline, that of confined car racing, which takes place on smooth asphalt surfaces in purpose-built race tracks. 

\begin{figure}[!htb]
	\centering
	\includegraphics[height=7cm]{images/cornerraceline}
	\caption{Example of confined car racing circuit}
	\label{fig:circuit-overhead}
\end{figure}

\begin{figure}[!htb]
	\centering
	\includegraphics[height=7cm]{images/cornerraceline}
	\caption{Example of racing line, straight and corners}
	\label{fig:circuit-breakdown}
\end{figure}

\begin{figure}[!htb]
	\centering
	\includegraphics[height=7cm]{images/cornerraceline}
	\caption{Forces acting on a car}
	\label{fig:forces-car}
\end{figure}

\textbf{Figures: Circuit overhead, Racing line overhead, Segmented by straight and corner}

\subsection{Racing Line}
A race driver needs to figure out how to achieve the shortest lap time for a given track.
\emph{A race driver needs to figure out how to go round a piece of asphalt in the minimum amount of time} \cite{GoingFaster}. In order to do so, he or she needs to develop techniques for more advanced vehicle control. One such technique is that of mastering the racing line (see Figure \ref{fig:circuit-breakdown}), which is considered the fundamental skill a race driver must understand and master before moving on to anything else \cite{GoingFaster}. The racing line is the best path through a circuit: if followed, it is the path that yields the shortest time at the highest average speed \cite{beckman1991physics}. The trickiest part of the racing line to master is that which overlaps circuit corner segments (see Figure \ref{fig:circuit-breakdown}). There are two aspects to mastery of the racing line: first, one has to identifying the path which should be taken, and secondly, one must stay on that path. In the first instance, one has to be able to visualise the racing line, while in the latter one has to control the car such that it stays on the line whilst achieving the highest possible average speed. Once the driver can visualise the racing line, he must further partition it, at and near a corner, in three sections. The first section is the breaking part, where the car needs to sufficiently decelerate in preparation for the corner. Braking is usually carried out in a straight line, ending right before the \emph{turn-in point}. The turn-in point refers to a point on the racing line where steering input is applied, forcing the car to turn into the corner. This action should be carried out smoothly, without jerking motions, taking the car all through the corner without too much correction to the steering. Smooth cornering prevents any abrupt changes to the g-forces and centre of gravity of the car (see Figure \ref{fig:forces-car}), which would result in unpredictable car behaviour \cite{GoingFaster}. Thus, the second partition of the racing line at a corner is the segment between the turn-in point and the apex point, which is the inside mid-point of the corner (see Figure \ref{fig:CornerRaceLine}). After the turn-in point, the driver aims for the apex point. The final section of the racing line in a corner lies from the apex point onwards, where the driver must gradually accelerate out of the corner, while still turning, aiming for the outside apex (see Figure \ref{fig:to add}.

\begin{figure}[!htb]
	\centering
	\includegraphics[height=7cm]{images/cornerraceline}
	\caption{Racing line through a 90" right corner}
	\label{fig:CornerRaceLine}
\end{figure}

As the driver gets acquainted to the racing line, usually at sub-optimal speeds, he must find the limit of the car, which is the highest speed the car can be driven while still retaining some measure of control. Various studies have been carried out to define such a limit in terms of the physical properties of the car and its environment \cite{beckman1991physics}. The most important property is the level of grip the car can achieve and sustain on track. A number of factors contribute to the level of grip. Most notably, one very important factor is the tires as they are the only contact the car makes with the track, and allow for braking, accelerating and turning forces to be transferred to the asphalt. 

Each tire has two properties which are of particular interest: the slip angle and slip ratio (see Figure \ref{fig:slipangle}). The slip angle is the angle between the tire's desired direction (perpendicular to the axis of rotation of the tire) and the tire's actual direction (the direction the car is moving in). Given both the actual direction of travel $\mathbf{D}_t$) and the desired direction ($\mathbf{D}_d$) are known, the slip angle $s_a$ is calculated as follows:
\begin{equation}
	s_a = \cos^{-1}(\hat{\mathbf{D}}_d \cdot \hat{\mathbf{D}_t}),
\end{equation}
\noindent where $\hat{\mathbf{D}_d} = \frac{\mathbf{D}_d}{|\mathbf{D}_d|}$ and $\hat{\mathbf{D}_t} = \frac{\mathbf{D}_t}{|\mathbf{D}_t|}$ are the normalised direction vectors for desired and travel directions respectively.

\begin{figure}[!htb]
	\centering
	\includegraphics[height=7cm]{images/slipangle}
	\caption{Slip Angle of a tire understeering while turning left}
	\label{fig:slipangle}
\end{figure}

\subsection{Cornering and braking}

Whenever the slip angle is above 0’ the tire is described as being in an understeering situation. Symptoms include Light steering, drifting towards the outside of a bend and possible tyre noise from the wheels. Assuming the tires are not damaged and the track is not wet nor dirty, understeer can be caused by active factors such as cornering speed throttle, braking, steering inputs and weight transfer. Other passive factors such as Weight distribution, drive layout, suspension and chassis setup, tyre type, wear and pressures also effect understeer. An understeer situation in a corner can be avoided by not entering too fast into a corner, not accelerating too aggressively in a corner, not braking through a corner, and not making any sudden changes which drastically upset the weight distribution of the car. Passive factors have to do with the way a car is mechanically setup, such factors will be taken in consideration during this project but will not be given great importance as this project aims to improve the drive’s skills. The project will focus on the active factors as these are the ones which the driver has direct input on while driving a car. It is known for a tire to have an optimal slip angle, this is the slip angle at which the tire can produce the most grip while cornering. A common road tire’s optimal slip angle is of 5’ while a slick tire which is purpose built for racing has an optimal slip angle of 8’-10’. These values may vary a bit depending on the tire brand\cite{beckman1991physics}

Moving on to oversteer, this is an other issue which can arise from lack of grip. Whereas understeer is caused by lack of grip in the front tires, oversteer is caused by lack of grip on the rear tires. Symptoms of oversteer include having the rear of the vehicle becoming unstable and 'light' and the car starts to rotate so the driver is facing towards the inside of the corner. Active factors causing oversteer are cornering speed, throttle, braking, steering inputs and weight transfer. The driver can avoid oversteer by not braking while in a corner and not accelerating too hard in a rear wheel drive as it makes the rear tires spin too fast, losing traction with the road.

\begin{figure}[!htb]
	\centering
	\includegraphics[height=7cm]{images/overundersteer}
	\caption{Visual representation for oversteer and understeer}
	\label{fig:slipangle}
\end{figure}

During acceleration and braking the tire experiences rotational forces, however these rotational forces do not match the expected velocity, this means at all time there is some level of slip occurring between the tire and the road beneath it. This slip is called slip ratio and is expressed in percentage. A slip percentage of 100\% would mean the tire is rotating, but the road is stationary, this is called a burnout or wheel spin. On the other hand, a percentage of -100\% would mean the tire is not rotating but the road beneath it is moving, this can occur while braking hard and is called locking the wheels\cite{pacejka2006tire}. While braking the driver must make sure not lock up the tires as this will cause the tires to wear out quicker while also drastically increasing the stopping distance. On the other hand braking too lightly will make the car take longer to decelerate with makes the driver lose time. In order to braking optimally the slip ratio should be between 10\% to 15\% \cite{GoingFaster}.

\subsection{Telemetry data}
Telemetry data is domain specialised data contains measurements and other data collected at remote or inaccessible points and are subsequently transmitted to receiving equipment for monitoring \cite{nasaTelemetry}. In motorsport telemetry data contains measurements from the engine and other vehicle dynamics. From these measurements one can the vehicle state at a particular point in time, such as speed, engine speed, temperatures, slip angles, slip ratios and so on. Telemetry is widely regarded as being the most important source of information by motorsport engineers as engineers can analyse this data and better under stand the car's and driver's strength and weaknesses \cite{CarDataAnalysis}.

\begin{figure}[!htb]
	\centering
	\includegraphics[height=7cm]{charts/telemetrydata.jpg}
	\caption{Visualised telemetry data from a Formula 1 team}
	\label{fig:telemetrydata}
\end{figure}

\section{Racing Rigs}

Moving on from the real world into the simulation world in which a racing rig is an integral part of achieving an authentic feel to the simulation experience. Racing rigs vary in price starting from a bundle with a steering wheel and pedal set costing less than €100 to ones used by professional racing teams costing thousands. The difference in price is partially due to built quality, but the biggest contributing factor is attributed to how well the rig mimics the real world. This is achieved by integrating force feedback, butt kickers and hydraulic pistons. Racing rigs can be categorised by four price range brackets. Entry level refer to the cheapest price range, racing rigs in the range offer the basics to get some one up and running.

The most basic racing rig is one which only has Steering wheel, than more sophisticated ones are made by adding pedals, shifters, racing seat, mounting frames and hydraulic pistons. Butt kickers and hydraulic pistons are commonly integrated in high end rigs such as ones built by Vesaro. Force feedback is a form of haptic technology which used to replicate the forces which are transferred through a steering wheel in a car onto the driver\cite{li2015can}. Butt kickers and hydraulic used to simulate lateral and longitudinal forces which a race driver is exposed to during racing.

\begin{figure}[!htb]
	\centering
	\includegraphics[height=7cm]{images/proracingrig.png}
	\caption{Professional racing rig by CXC Simulations}
	\label{fig:slipangle}
\end{figure}







