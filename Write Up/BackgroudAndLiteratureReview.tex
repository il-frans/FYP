\section{Background Work}

Backgroud
This section may describe such things as:
• the wider context of the project,
• the anticipated benefits of the system,
• the likely users of the system,
• any theory associated with the project,
• the software/hardware development method(s) used,
• any special diagramming conventions used, 13
• existing software (or hardware) that is relevant to the system, 

LR
• An extensive study in the area of interested, highlighting the strengths and weaknesses of
existing methods.
• A review of the state-of-the-art published material in the area.
• A summarization of the published material in the area.
• A critical analysis of exiting material and methods.
• An explanation showing why the literature chosen to review is relevant to the FYP. 

\subsection{Motorsport racing}

In sports individuals or groups compete to be first to achieve a particular objective. In the case of circuit motorsport races, in which motorised vehicles go round a course. Each racing discipline or series has its own rules. However, at the core, all disciplines participants aim to complete a full lap of the circuit in the least amount of time. Some disciplines focus on achieving one fast lap, such as time trials, while others focus on achieving the least amount of time across a fixed amount of laps, such as FIA's Formula 1 series. This dissertation will focus on confined car racing taking place on smooth asphalt surfaces in purpose built race tracks. 

A race driver needs to figure out how to go round a piece of asphalt in the minimum amount of time \cite{GoingFaster}. In order to do so, he or she needs to develop techniques for more advanced vehicle control. One such technique is that of mastering the race line, which is considered the the fundamental skill a race driver must understand and master before moving on to anything else \cite{GoingFaster}. The racing line is the best path through a circuit, it is the one which takes the least time while keeping the higher average speed \cite{beckman1991physics}. The trickiest part of the racing line to master is that of a corner. This is split into two parts, identifying the line which should be taken and staying on the line. The first part refers to being able to visualise the racing line while the later refers to actually being able to control the car so that it stays on the line. 

\subsection{Video games and Serious Games}

Baranowski et al \cite{yuserious} define games as a physical or mental contest with a goal or objective, played according to a framework, or rule, that determines what a player can or cannot do inside a game world. The definition covers the setup of a game, while a physical or mental contest, played according to specific rules, with the goal of amusing or rewarding the participant the reward aspect of games.

Video games are built on top of these core values with the addition of having the game world confined to some sort of digital medium. The first video game was created by William Higinbotham; it was a tennis game to be played on a television set\cite{stanton2015brief}. From the early days of video games, their main aim was always to provide some degree of entertainment. The entertainment value is achieved in various ways depending on the gaming platform, game genre and the target audience. Modern video games are simply made up of three fundamental components: story, art and software \cite{zyda2005visual}.

Moving on to serious games this type of games are considered a mix of simulation and game to improve eduction \cite{abt1970}. The idea behind a serious game is to connect a serious purpose to knowledge and technologies from the video game industry\cite{michael2005serious}. The boundaries of serious games are debated, mostly due to the fact that serious games attract multiple domains making it hard to come up with a common boundary. However, the common denominator across all domains seems to be serious game designers use people's interest in video games to capture their attention for a variety of purposes that go beyond pure entertainment\cite{djaouti2011classifying}.

The main contrast between video games and serious games is the use of pedagogic activities which aim to educate or instruct knowledge or skill \cite{zyda2005visual} in serious games as opposed to the pure leisurely aspects of the video game. Pedagogy is given preference over the amusement value which in some cases might not be found in serious games \cite{zyda2005visual}.

\subsection{Sim Racing as a Serious Game}

Simulation racing games (sim racing) such as Asseto Corsa \cite{assestoCorsa} and Project CARS \cite{ProjectCars} , which are off-the-shelf products, provide a sim racing experience within budget for the average video game consumer. The aim with such games is to replicate real life cars, race car dynamics and track locations to amuse and entertain the player. The challenge aspect is achieved by pitting the user against other computer drivers known as AI players, or in multiplayer online races, which are played against other human players. In some cases, a user can compete against oneself by taking on a ghost - a recording of the player's best lap for a particular track. Sim racing the definition of what a video game is however, they miss the pedagogy activities which would qualify them as serious games. Most of the modern sim racing games do aid the player to improve by means of implementing aids. Such aids might include showing the racing line while also highlighting the braking and acceleration points. Other aids include anti lock brakes, traction control and stability control, these are implemented in a passive way. With the exception of the racing line, the player is not told when and what is being done wrong. This results in users having to figure out their own mistakes by means of practicing without any guidance or feedback from with the game. This final year project aims to implement a module which is plugged into an off the shelf racing simulator which. This module trains users by letting them know what is being done wrong, when it's being done wrong and most importantly how to avoid making the same mistake.