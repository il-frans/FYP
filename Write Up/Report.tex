\documentclass{report}

\usepackage{color}
\usepackage{cite}
\usepackage{float}
\usepackage{xspace}
\usepackage{amsfonts}
\usepackage{graphicx}
\usepackage{pgfgantt}
\usepackage{setspace}
\usepackage{gensymb}
\usepackage{pdfpages}
\usepackage{eurosym}
\usepackage{booktabs}
\usepackage{hyperref}
\usepackage[toc,page]{appendix}
\usepackage{multirow}

\hypersetup{
	colorlinks,
	citecolor=black,
	filecolor=black,
	linkcolor=black,
	urlcolor=black
}

\newcommand*{\figuretitle}[1]{%
	{\centering%   <--------  will only affect the title because of the grouping (by the
		\textbf{#1}%              braces before \centering and behind \medskip). If you remove
		\par\medskip}%            these braces the whole body of a {figure} env will be centered.
}

\linespread{1.5}
\begin{document}

\begin{titlepage}
	\begin{center}
		\begin{spacing}{1.5}
			\huge{\textbf{Telemetry-based Optimisation for User Training in Racing Simulators}}
		\end{spacing}
		
		\LARGE{Fran\c{c}ois Buhagiar
		
		Supervisor : Dr Keith Bugeja\\
		Co-supervisor : Dr Sandro Spina}
		
		\vspace{5mm}
		\includegraphics[scale=5]{images/UOM_Logo}
		
		\vfill
		\begin{spacing}{1.5}
			\Large{\textbf{Department of Computer Information Systems\\
				University of Malta\\
				March 2016 }}
		\end{spacing}
		\vspace{10mm}
		\normalsize  Submitted in partial fulfilment of the requirements for the degree of B.Sc. I.T. (Hons.) in Software Development
		\clearpage
	\end{center}
\end{titlepage}

\pagenumbering{roman} 

\newpage
\includepdf[pages=-]{FYPSubmissionSizeDeclaration.pdf}
\includepdf[pages=-]{authenticityformug.pdf}

\newpage
\begin{abstract}

This study explores the research, design, development, implementation and evaluation of a novel real-time telemetry aided system for racing driving training for non-experts. It does so by starting off exploring what serious games are and their applications, what motorsport is, what is expected of a race driver and what's the state of the art of virtual sim racing. After which a system for real time feedback is designed which provides auditory feedback in the form of descriptive speech while a user makes use of a racing rig and a sim racing game. User studies are carried out to evaluate the effectiveness of the system.  Evaluation is based on the collected telemetry data and questionnaires filled out by participants. Initial results are encouraging, indicating that there is potential for performance gain and driver confidence enhancement based on the audio feedback.


\end{abstract}

\newpage
\section*{Acknowledgements}

I am extremely thankful and indebted to Dr Keith Bugeja and Dr Sandro Spina for sharing their expertise, and sincere and valuable guidance and time, and encouragement during this work. Their systematic approach and eye for detail have allowed this project to became of much higher quality than it would have otherwise have been. 

I wish to express my sincere thanks to Luke Chircop, for providing the required sim rig hardware for this project. Without his contribution it would have been much harder for this project to source the hardware for the experiments.

\newpage
\tableofcontents

\newpage
\listoffigures
\listoftables

\newpage
\pagenumbering{arabic} 
\setcounter{page}{1}
\chapter{Introduction}

% Intro to the areas
The gamification of areas of activity such as marketing, problem solving and education \cite{michael2005serious} has validated the use of serious games beyond their initial military use in training strategic skills \cite{djaouti2011classifying}.  Serious games simulate real-world processes designed for the purpose of solving a problem, making their main purpose that of training or educating users. Their popularity has been steadily increasing, as has their adoption, with military \cite{djaouti2011classifying} and emergency service providers (e.g. firefighters \cite{michael2005serious}) employing them to train for specific scenarios that might be encountered on the respective jobs. Motorsports cover a broad range of activities and vehicles, and as with all major forms of sporting activities, require training and dedication, with a pedagogic aspect arising in rote learning and mentoring by experts. The arenas in which motorsport events take places are called circuits; there is a large selection of the latter, ranging from purposely built race tracks to public roads to natural formations such as hills and quarries. There is also a diverse selection of vehicles that take part in motorsports, with the greatest demarcation existing between motorbikes and cars. The focus of this dissertation is that of unifying serious games and motorsport racing; specifically, it will try to show whether a serious game is a powerful enough pedagogical tool that can be used to tangibly improve the performance of race drivers. The scope of the project is limited to four-wheeled cars racing on purposely-built confined circuits with a smooth tarmac surface.  

% How the project will be tackled
This study will take an exploratory and descriptive research approach by first looking into the equipment and environment race drivers work with and their properties, followed by identifying what skills race drivers must master, why these skills are important and the impact they have on their performance. Sim racing hardware will be defined and hardware from various price ranges will be explored. Academic literature will be consulted particularly on what makes a serious game, their applications and state of the art within the domain of motorsport and driving in general. Possible training mechanism will be investigated, aiming to develop an artefact which given telemetry data, it is able to identify erroneous driving in the context of racing and manage to convey to the user what should be done in order to avoid such behaviour.  Evaluation strategies will be investigated and one is to be implemented in order to be able to quantify the effectiveness of the artefact. Finally, findings from the evaluation strategy will be presented and any claims derived from the findings will be put forward.

\section{Aims and Objectives}
\section{Dissertation Overview}

\newpage
\section{Background}

\subsection{Motorsport racing}

In sports individuals or groups compete to be first to achieve a particular objective. In the case of circuit motorsport races, in which motorised vehicles go round a course. Each racing discipline or series has its own rules. However, at the core, all disciplines participants aim to complete a full lap of the circuit in the least amount of time. Some disciplines focus on achieving one fast lap, such as time trials, while others focus on achieving the least amount of time across a fixed amount of laps, such as FIA's Formula 1 series. This dissertation will focus on confined car racing taking place on smooth asphalt surfaces in purpose built race tracks. 

\subsubsection{Racing Line}

A race driver needs to figure out how to go round a piece of asphalt in the minimum amount of time \cite{GoingFaster}. In order to do so, he or she needs to develop techniques for more advanced vehicle control. One such technique is that of mastering the race line, which is considered the the fundamental skill a race driver must understand and master before moving on to anything else \cite{GoingFaster}. The racing line is the best path through a circuit, it is the path which takes the least time while keeping the higher average speed \cite{beckman1991physics}. The trickiest part of the racing line to master is that of a corner, this task is  split into two parts, identifying the line which should be taken and staying on the line. The first part refers to being able to visualise the racing line while the later refers to actually being able to control the car so that it stays on the line. Once the driver has an idea of the race line which the corner should be taken at, he or she must further split the line into three sections. The first part is the braking part, during this part the driver needs to decelerate in preparation for the corner, this is usually carried out in a straight line and end right before the turn in point. The turn in point refers to a point on the corner race line in which steering input is applied, allowing the car to turn into the corner. The turn in motion should be a smooth one, taking the car all through the corner without having to do too much corrections to the steering. Being smooth while cornering allows the G-forces and centre of gravity of the car to not be abruptly effected which would result in unpredictable car behaviour \cite{GoingFaster}. The aim of the turn in section is to aim for the apex point, this is the inside middle point of the corner. The final part of tackling a corner is the acceleration part, after the apex point, the driver must start to gradually accelerate out of the corner while still turning out of the corner aiming for the outside apex.

\begin{figure}[!htb]
	\centering
	\includegraphics[height=7cm]{images/cornerraceline}
	\caption{Race line through a 90" right corner}
	\label{fig:CornerRaceLine}
\end{figure}

After the driver manages to drive the race line at a relatively slow speed, the driver must find the limit of the car. This is the maximum speed the car can be driven while still allowing the driver to have maximum control over the car. Various studies have been carried out to define such a limit in terms of the physical properties of the car and environment around it. The most important property is the level of grip the car can achieve and sustain on track. Various factors contribute to the level of grip, most notable are the tires which the car is being driven on, as the tires are the only actual contact to the track, allowing for braking, accelerating and turning forces to be transferred to the asphalt\cite{beckman1991physics}.

Each tire has two properties which are of particular interest to a drive, the slip angle and slip ratio. The slip angle is the angle between a tire’s direction of travel and the actual direction the tire is going towards. Given both the actual direction of travel and desired direction of travel are known, it becomes trivial to calculate the angle which is done by calculating the arch of the two vectors as shown in \ref{fig:slipangle}.

\begin{figure}[!htb]
	\centering
	\includegraphics[height=7cm]{images/slipangle}
	\caption{Slip Angle of a tire understeering while turning left}
	\label{fig:slipangle}
\end{figure}

\subsubsection{Cornering and braking}

Whenever the slip angle is above 0’ the tire is described as being in an understeering situation. Symptoms include Light steering, drifting towards the outside of a bend and possible tyre noise from the wheels. Assuming the tires are not damaged and the track is not wet nor dirty, understeer can be caused by active factors such as cornering speed throttle, braking, steering inputs and weight transfer. Other passive factors such as Weight distribution, drive layout, suspension and chassis setup, tyre type, wear and pressures also effect understeer. An understeer situation in a corner can be avoided by not entering too fast into a corner, not accelerating too aggressively in a corner, not braking through a corner, and not making any sudden changes which drastically upset the weight distribution of the car. Passive factors have to do with the way a car is mechanically setup, such factors will be taken in consideration during this project but will not be given great importance as this project aims to improve the drive’s skills. The project will focus on the active factors as these are the ones which the driver has direct input on while driving a car. It is known for a tire to have an optimal slip angle, this is the slip angle at which the tire can produce the most grip while cornering. A common road tire’s optimal slip angle is of 5’ while a slick tire which is purpose built for racing has an optimal slip angle of 8’-10’. These values may vary a bit depending on the tire brand\cite{beckman1991physics}

Moving on to oversteer, this is an other issue which can arise from lack of grip. Whereas understeer is caused by lack of grip in the front tires, oversteer is caused by lack of grip on the rear tires. Symptoms of oversteer include having the rear of the vehicle becoming unstable and 'light' and the car starts to rotate so the driver is facing towards the inside of the corner. Active factors causing oversteer are cornering speed, throttle, braking, steering inputs and weight transfer. The driver can avoid oversteer by not braking while in a corner and not accelerating too hard in a rear wheel drive as it makes the rear tires spin too fast, losing traction with the road.

\begin{figure}[!htb]
	\centering
	\includegraphics[height=7cm]{images/overundersteer}
	\caption{Visual representation for oversteer and understeer}
	\label{fig:slipangle}
\end{figure}

During acceleration and braking the tire experiences rotational forces, however these rotational forces do not match the expected velocity, this means at all time there is some level of slip occurring between the tire and the road beneath it. This slip is called slip ratio and is expressed in percentage. A slip percentage of 100\% would mean the tire is rotating, but the road is stationary, this is called a burnout or wheel spin. On the other hand, a percentage of -100\% would mean the tire is not rotating but the road beneath it is moving, this can occur while braking hard and is called locking the wheels\cite{pacejka2006tire}. While braking the driver must make sure not lock up the tires as this will cause the tires to wear out quicker while also drastically increasing the stopping distance. On the other hand braking too lightly will make the car take longer to decelerate with makes the driver lose time. In order to braking optimally the slip ratio should be between 10\% to 15\% \cite{GoingFaster}.

\subsection{Racing Rigs}

Moving on from the real world into the simulation world in which a racing rig is an integral part of achieving an authentic feel to the simulation experience. Racing rigs vary in price starting from a bundle with a steering wheel and pedal set costing less than €100 to ones used by professional racing teams costing thousands. The difference in price is partially due to built quality, but the biggest contributing factor is attributed to how well the rig mimics the real world. This is achieved by integrating force feedback, butt kickers and hydraulic pistons. Racing rigs can be categorised by four price range brackets. Entry level refer to the cheapest price range, racing rigs in the range offer the basics to get some one up and running.

The most basic racing rig is one which only has Steering wheel, than more sophisticated ones are made by adding pedals, shifters, racing seat, mounting frames and hydraulic pistons. Butt kickers and hydraulic pistons are commonly integrated in high end rigs such as ones built by Vesaro. Force feedback is a form of haptic technology which used to replicate the forces which are transferred through a steering wheel in a car onto the driver\cite{li2015can}. Butt kickers and hydraulic used to simulate lateral and longitudinal forces which a race driver is exposed to during racing.

\begin{figure}[!htb]
	\centering
	\includegraphics[height=7cm]{images/proracingrig}
	\caption{Professional racing rig by Vesaro}
	\label{fig:slipangle}
\end{figure}









\newpage
\chapter{Literature Review}
\label{chp-LiteratureReview}
\label{chp:literature-review}

This chapter provides a brief exposition of the state of the art and related work in race driving simulation. The chapter starts by drawing a distinction between video games and serious games, and motivates this demarcation. This is followed by a discussion of the pedagogical aspect of serious games. Finally, an overview of simulation racing is given, with a focus on the genre's intersection between entertainment and pedagogical factors.

\section{Video Games and Serious Games}
Baranowski et al \cite{yuserious} define games as a physical or mental contest with a goal or objective, played according to a framework, or rule, that determines what a player can or cannot do inside a game world. The definition covers the setup of a game, while a physical or mental contest, played according to specific rules, with the goal of amusing or rewarding the participant the reward aspect of games.

Video games are built on top of these core values with the addition of having the game world confined to some sort of digital medium. The first video game was created by William Higinbotham; it was a tennis game to be played on a television set\cite{stanton2015brief}. From the early days of video games, their main aim was always to provide some degree of entertainment. The entertainment value is achieved in various ways depending on the gaming platform, game genre and the target audience. Modern video games are simply made up of three fundamental components: story, art and software \cite{zyda2005visual}.

Serious games are considered a mixture of simulation and video game to provide some form of education\cite{abt1970}. The idea behind serious games is to connect a serious purpose to knowledge and technologies from the video game industry\cite{michael2005serious}. The boundaries of serious games are debated, mostly due to the fact that they attract multiple domains making it hard to come up with a common boundary. However, the common denominator across all domains seems to be serious game designers use people's interest in video games to capture their attention for a variety of purposes that go beyond pure entertainment\cite{djaouti2011classifying}.

The main difference between video games and serious games is the use of pedagogic activities which aim to educate or instruct in the latter, as opposed to the purely leisurely aspects of video games\cite{zyda2005visual}. Pedagogy is given preference over the amusement value which in some cases might be missing in serious games \cite{zyda2005visual}. All serious games involve some form of learning; they educate the player with a specific type of content, whereas video games simply aim to entertain the player\cite{Harteveld2007}. Since the main objective of video game designers is to make the game fun, the content and controls should be at the service of making the game entertaining. On the other hand, serious game designers have multiple objectives; they need to create a compelling (and possibly fun) game, but which fulfils some ulterior goal such as educating the player.  From this it follows that three aspects are essential for a serious game: fun, learning and validity\cite{Harteveld2007}.

\subsection{Pedagogy}
In order to produce a valid pedagogical experience, learning objectives, target groups and challenges needs to be clearly identified before designing a serious game\cite{moser2002methodology}. Various pedagogical theories exist which can be applied to a serious game, some of which are behaviourism, cognitivism, constructivism and situated learning\cite{egenfeldt2005beyond}; a number of properties can be synthesised from these theories: 

\begin{description}
\item[Experience] Games tend to provide learning-by-doing. Many games make use of pop-up windows with extensive amount of text that are supposed to have educational value. This technique could provide too much information, time pressure or other factors inside a game environment which could potentially lead to cognitive overload or lead a person to filtering out critical information\cite{egenfeldt2005beyond}.

\item[Exploration] An important property of a game is that of requiring an active, participative attitude of the learner. The game world, including rules, mechanics and environment need to be explored and discovered by the learner. Many poorly designed games force the player to do something, while they should just let the player figure it out or at least guide the player into doing so.

\item[Incremental] The learning process should occur incrementally as it will otherwise prove too demanding for a player; that is the way the human brain functions\cite{moser2002methodology}.
\end{description}

There are various channel or media through which teaching and learning can occur; these channels are constrained by the capabilities and limitations of the target user demographic. The three main channels considered are visual, auditory and kinaesthetic. Some instructions might be better conveyed through visual cues, while other instructions work better when communicated through auditory or kinaesthetic means. Leahy et al. show that a mix of media or channels work better, with one complementing the other\cite{leahy2003auditory}. For instance, in cases where timing is a factor, having a visual image further complemented with an audio stream or vice versa may reinforce knowledge acquisition in the learner. A further consideration has to be made when applying this to the vehicle driving context; it is important avoid or at least mitigate the effect channels might have on the concentration of the driver\cite{engstrom2005effects}. 

\section{Racing Simulation Games}
Racing simulation games (sim racing) such as Asseto Corsa~\cite{assestoCorsa} and Project CARS~\cite{ProjectCars}, which are off-the-shelf products, provide a sim racing experience within budget for the average video game consumer. The aim with such games is to replicate real life cars, race car dynamics and track locations to amuse and entertain the player. The challenge aspect is achieved by pitting the user against other computer drivers known as AI players, or in multiplayer online races, which are played against other human players. In some cases, a user can compete against oneself by taking on a ghost - a recording of the player's best lap for a particular track. Sim racing the definition of what a video game is however, they miss the pedagogy activities which would qualify them as serious games. Most of the modern sim racing games do aid the player to improve by means of implementing aids. Such aids might include showing the racing line while also highlighting the braking and acceleration points. Other aids include anti-lock brakes, traction control and stability control, these are implemented in a passive way. With the exception of the racing line, the player is not told when and what is being done wrong. This results in users having to figure out their own mistakes by means of practicing without any guidance or feedback from with the game. This final year project aims to implement a module which is plugged into an off the shelf racing simulator which. This module trains users by letting them know what is being done wrong, when it's being done wrong and most importantly how to avoid making the same mistake. Further more this project builds on the premise put forward which shows that users are able to learn road driving skills into a virtual world and then successfully applying them to the real world\cite{li2015can, vogel2006computer}. Although studies have been carried out involving training for road drivers, none have looked into teaching on racing circuits with the aim of improving racing and car handling techniques.

\section{Summary}
This chapter has provided an overview of the literature in the area of serious games, with an emphasis on its pedagogical aspects. The difference between serious and video games was clearly delineated, with the properties of each discipline clearly outlined. The pedagogical aspect of serious games was reviewed in terms of its applicability to driving simulations, stressing how important it is for these to reproduce real-world car dynamics in the virtual world for any measure of success to be achieved.

\newpage
\chapter{Research Methodology}
\label{chp:methodology}
This chapter provides a detailed exposition of the methodology used throughout this work. This study has conducted exploratory and descriptive research to determine whether the use of simulation and an automated feedback driven system, a user can be trained as a race driver. The chapter is structured as follows: \S\ref{sec:meth-overview} provides a general overview of the overarching methodology used in the study, \S\ref{sec:meth-experiment-setup} describes in length the design of the instrument used to acquire experiment data and results, \S\ref{sec:meth-experiment-structure} presents the experimental procedure and the rationale behind it, \S\ref{sec:meth-data-gathering} identifies the information and data acquired through the experimental and descriptive methodologies employed, and finally \S\ref{sec:meth-data-analysis} presents the data analysis mechanisms employed to substantiate our conclusions.
%
% Overview
%
\section{Overview}
\label{sec:meth-overview}
This study conducted experimental and descriptive research on the viability of the use of simulators in conjunction with an automated feedback system for improving race driving skills in the normal population. Specifically, a user study was devised and carried out with the primary goals being:
\begin{enumerate}
	\item To determine whether a context-based feedback system can improve the skills of a participant
	\item To quantify the magnitude of this improvement, if (1) is true
\end{enumerate}

These goals were addressed by means of an experimental setup based on a race-driving simulator, using which objective measurement of the participant performance could be gathered and analysed, and a questionnaire for relating the participant's experience with the experimentally-gathered data. The setup of the experiment is explored in more detail in \S~\ref{sec:meth-experiment-setup}.

Since the main goal of this study is that of assessing how effective a feedback system is in the learning process, the independent variable in the experiment is the ability to receive feedback. The hypothesis is that participant performance (such as average lap time) is improved through feedback, and is thus a dependent variable. However, practising without feedback can also lead to changes in the dependent variable; therefore this is controlled for by having two groups of participants: the experimental group that receives feedback and the control group that doesn't. Random assignment is used to determine a participant's group.

A questionnaire, to be administered to the participants at the end of the session, will be designed to help normalise and control for other factors that may influence dependent variables, and hence, the outcome of the experiment. The design of the questionnaire also helps in bridging the participants' perception of their performance with the actual performance data, possibly providing further insight into the results. A questionnaire was preferred to an interview because it is easier to administer, it lends itself to group administration and also allows confidentiality \cite{introductiontobehavioralresearchmethods}.. It is indeed true that interviews permit a greater freedom of expression on behalf of the participant; however, questionnaires create a sense of anonymity that encourages the participants to be more truthful in their answers \cite{introductiontobehavioralresearchmethods}.

\section{Experiment Design}
\label{sec:meth-experiment-design}
In the experiment, each group would be utilising the same car and racetrack. Bastow et al. \cite{bastow2004car} suggest that cars equipped with a front wheel drivetrain may be easier to handle. The Fiat 500 Abarth was the car chosen for the experiments, partially based on Bastow et al.'s findings. The car is relatively low-powered and thus, easier to use by beginning drivers. The Silverstone National race track has the desirable properties of being flat and smooth, without uneven surfaces or bumps which may result in loss of control in rookie drivers. Furthermore, the way the track is structured, with wide run-off areas located along the circuit where drivers are most likely to lose control of the car, allows the car to slow down before colliding with barriers or other stationary objects.

Two feedback mechanisms have been considered for this experiment, \emph{visual}, through the use of a heads-up-display (HUD) superimposed on the simulation display, or \emph{auditory}, by means of descriptive speech projected through loud speakers. Leahy et al. \cite{leahy2003auditory} argue that auditory feedback is less intrusive than visual clues; based on these findings, it was decided that the system should provide feedback using auditory clues.

\begin{figure}
	\centering
	\includegraphics[width=\textwidth]{images/experiment-setup-schematic.png}
	\caption[Experiment Setup Schematic]{Top-down and side views of experiment setup}
	a = 45cm, b = 75cm, c = 36cm, d = 120cm, e = 20cm, f = 30cm, g = 35cm
	\label{fig:meth-experiment-setup}
\end{figure}

\section{Experiment Materials}
\label{sec:meth-experiment-setup}
The setup of the experiment was divided into three material categories: \emph{simulation environment}, \emph{simulation hardware} and \emph{simulation software}, with each category subscribing to a number of desirable properties: 

\begin{description}
	\item [Environment] The experiment should be carried out in an isolated, noise-free and well-lit room. Participants would be let in the room one at a time, to ensure the experiment is conducted without any distractions.
	\item [Hardware] The hardware components identified for this experiment are the (i) display output, (ii) audio output, (iii) steering wheel, (iv) gear shifter, (v) acceleration, brake and clutch pedals and (vi) seating frame.
	\begin{description}
		\item [Display and Audio] A 32"+ display capable of outputting graphics at a progressive resolution of 1080 pixels vertically at 60 Hz (1080p60) is required; smaller display sizes would not yield a large enough solid angle for the participant to feel immersed in the simulator. For audio output, a stereo setup with a speaker output power of 10W should be sufficient; at a distance of 1m, these speakers can generate sound volume up to 100db, where normal conversation is around 60db.
		\item [Driving Controls] The driving controls include (ii)-(iv); minimally a steering wheel should provide the same number of revolutions as a racing car, providing accurate force feedback to let the participants accurately assess the behaviour of the car. Gear shifters do not implement feedback mechanisms; however, given their ubiquity, H-shifters are preferred since they are the kind most drivers are familiar with. High-end pedal systems use hydraulics to simulate the variability of force required on part of the driver to actuate a pedal during different stages.
		\item [Seating Frame] The seating frame should ensure a seating position akin to a driver in a racing car. The seating frame should also provide the ability to move the seat back and forward in order for participants to be able to seat comfortable and be able to reach the pedals.
	\end{description}
	\item [Software] The software aspect of the experiment, which is primarily the racing simulator, should have a number of desirable properties. In particular, the simulator should provide (i) a realistic driving model, (ii) driving aid customisation, (iii) high-fidelity graphics, (iv) real-world tarmac circuits, (v) and telemetry data access through an interface that is intuitive to use.
	\begin{description}
		\item [Realistic Driving Model] For this experiment, a realistic driving model is a sine qua non. Recreating real-world conditions requires a high-fidelity physics simulation of race car dynamics that is as close to the real thing as possible.
		\item [Driving Aid Customisation] Required to control experiment variables, such as tyre wear and tear. The aim is to prevent car behaviour from changing as the experiment progresses. 
		\item [High-fidelity Graphics] Required to provide the participant with a fully immersive experience.
		\item [Real-world Tarmac Circuits] Most racing simulators accurately replicate real-world tracks, up to minute details such as elevation changes and bumps. This level of realism exposes participants to real-life scenarios. Furthermore, the simulator should provide circuits that loop, to streamline experiments and remove the need to reset the software in order to drive another lap.  
		\item [Telemetry Data Access] The premise of this project hinges on the ability to read telemetry data from a simulator. Thus, any software that does not provide this feature is a priori discarded. However, easier and cleaner interfaces for reading telemetry data, and the breadth of telemetry data offered, are a deciding factor in the choice of racing simulator to use.
	\end{description}
\end{description}

\subsection{Hardware Selection}
A number of racing simulator input devices have been considered (see Table \ref{tab:racing-input}). The choice was narrowed down to the most popular devices among racing game enthusiasts, the Logitech G25~\cite{logitecg25}, the Thrustmaster TX~\cite{TrustmasterTX} and the Fanatec CS~\cite{Fanatec}. The G25 provides the minimally required features: a steering wheel with a $900\degree$ turn, 3-pedal set, H-shifter and force feedback; furthermore, the G25 is also a very affordable device. Both the Thrustmaster and Fanatec are high-end devices, which explains the cost difference. Furthermore, neither come with the 3-pedal set and the H shifter, which have to be purchased separately. The Logitech G25 was chosen for being the most cost-effective of the three options.

\begin{table}[htb!]
    \centering
    \begin{tabular}{lccccc}
        \toprule
        \textbf{Device} & \textbf{$900\degree$} & \textbf{3-Pedal Set} & \textbf{H-Shifter} & \textbf{Feedback} & \textbf{Cost} \\
        \midrule
        Logitech G25 & \checkmark & \checkmark & \checkmark & Low & 240 \\
        Thrustmaster TX & \checkmark & & & Med & 324 \\
        Fanatec CS & \checkmark & & & High & 850 \\
		\bottomrule		
	\end{tabular}
	\caption{Comparison of input devices}
	\label{tab:racing-input}	
\end{table}

\subsection{Software selection}
Similarly to the hardware selection process, a number of racing simulators have been considered (see Table \ref{tab:sim-choice}). Although there is a myriad of racing simulators available, the selection process narrowed them down to a popular handful: Forza Motorsports 6 \cite{forza} by Turn 10 Studios, Project Cars\cite{ProjectCars} by Slightly Mad Studios, Assetto Corsa\cite{assestoCorsa} by Kunos Simulazioni, iRacing\cite{iRacing} by iRacing.com Motorsports Simulations and Dirt \cite{dirtgame} by Codemasters. Unfortunately, notwithstanding its quality, Forza Motorsports 6 does not provide access to telemetry information. Dirt also suffers from the same shortcoming, in addition to providing very little in terms of track applicability: very few tracks are circuit-based. iRacing provides most of the functionality required by the experiment, however in terms of visual quality it suffers when compared to Project Cars and Assetto Corsa. Finally, the choice between Assetto Corsa and Project Cars was made on the basis of ease of interfacing for the acquisition of telemetry data; here we felt Assetto Corsa provided a clean, intuitive and ultimately superior interface to telemetry acquistion via User Datagram Protocol (UDP) connections.

%The comparison table shows the desired properties a sim game must have, and a selection of games which have been considered. From the start it is clear, Dirt and Forza 6 should be discarded as none provide telemetry data to be read. Assetto Corsa, Project Cars and iRacing are closely matched, however Assetto Corsa has the leading edge as it provides an easy way to interface via UDP, well document and wide developer community who can help. Furthermore Assetto Corsa provides good visual graphics. For these reasons Assetto Corsa has been picked as the sim to be used.

\begin{table}[htb!]
    \centering
    \begin{tabular}{lcccccc}
        \toprule
        \textbf{Simulator} & 
        \multicolumn{1}{p{1.2cm}}{\centering \textbf{Driving} \\ \textbf{Model}} &
        \multicolumn{1}{p{1.2cm}}{\centering \textbf{Driving} \\ \textbf{Aids}} &
        \multicolumn{1}{p{1.2cm}}{\centering \textbf{Visual} \\ \textbf{Quality}} &
        \multicolumn{1}{p{1.1cm}}{\centering \textbf{Tracks}} &
        \multicolumn{1}{p{1.1cm}}{\centering \textbf{Tele-metry}} &
        \multicolumn{1}{p{1.1cm}}{\centering \textbf{I/F Ease}} \\
        \midrule
        Forza & \checkmark & \checkmark & \checkmark & \checkmark & & \\ 
        Project Cars & \checkmark & \checkmark & \checkmark & \checkmark & \checkmark & \\ 
        Assetto Corsa & \checkmark & \checkmark & \checkmark & \checkmark & \checkmark & \checkmark \\ 
        iRacing & \checkmark & \checkmark & & \checkmark & \checkmark & \checkmark \\ 
        Dirt & \checkmark & \checkmark & \checkmark & & & \\ 
		\bottomrule		
	\end{tabular}
	\caption[Comparison of Racing Simulators]{Comparison of racing simulators on the basis of a realistic driving model, customisable driving aids, high-fidelity graphics quality, applicability of tracks, availability of telemetry information, and ease of interfacing.}
	\label{tab:sim-choice}	
\end{table}

%
%\begin{center}
%	\begin{tabular}{ | l | l | l | l | l | l |}
%		\hline
%			& Assetto Corsa\cite{assestoCorsa} & pCars \cite{ProjectCars} & iRacing \cite{iRacing} & Dirt \cite{dirtgame} & Forza 6 \cite{forza} \\ \hline
%		Realistic model	& \checkmark &\checkmark & \checkmark & \checkmark & \checkmark \\ \hline
%		Tracks	& \checkmark &\checkmark & \checkmark &  & \checkmark \\ \hline
%		Disable wear & \checkmark & \checkmark & \checkmark & \checkmark & \checkmark \\ \hline
%		Telemetry data	& \checkmark & \checkmark & \checkmark &  &  \\ \hline
%		Ease of Interfacing	& \checkmark &  & \checkmark &  &  \\ \hline
%		Quality of graphics & \checkmark & \checkmark &  & \checkmark & \checkmark \\ \hline
%	\end{tabular}
%\end{center}

\section{Participant Information and Feedback}
Questionnaires are a very important tool for acquiring insight from the point of view of experiment participants, and when compared to interviews, they provide a framework within which respondents can answer more truthfully due to lack of social external pressures. Leary et al. \cite{introductiontobehavioralresearchmethods} provide a set of guidelines for compiling questionnaires, consisting of seven general rules:
\begin{enumerate}
	\item Be specific and precise in phrasing the questions;
	\item Write the questions as simply as possible, avoiding difficult words, unnecessary jargon, and cumbersome phrases;
	\item Avoid making unwarranted assumptions about the respondents;
	\item Conditional information should precede the key idea of the question;
	\item Do not use double-barrelled questions;
	\item Pretest the questions.
\end{enumerate}
Responses from questionnaires designed using these guidelines are valid in the general case. A further challenge posed by questionnaire compilation is that of choosing a response format for each question: open-ended questions may serve to collect more information but be less conducive to analysis, while on the other hand, more restrictive and constrained response formats may lack expressivity but be easier to analyse. Leary et al. suggest using constrained response formats, such as the Likert scale \cite{likert1932technique}, when dealing with behaviours, thoughts or feelings that can vary in frequency or intensity, and open-ended questions in cases where further insight is desired. 

Two qualitative questionnaires have been designed in accord with these guidelines, one with the aim of gathering insight into the participant demographics, and a second to help normalise and control for factors that may influence dependent variables, to bridge the participants' perception of their performance with the actual execution, and to gather other insight and feedback about the experiment.

%A questionnaire, to be administered to the participants at the end of the session, will be designed to help normalise and control for other factors that may influence dependent variables, and hence, the outcome of the experiment. The design of the questionnaire also helps in bridging the participants' perception of their performance with the actual performance data, possibly providing further insight into the results.

%The challenge in choosing a response format in a question lies in identifying whether one should go for an open ended question in which more information might be collected, or on the other hand, use a rating scale response format, where the response is more constrained but may be easier to analyse. Leary et al. suggest using the former for questions dealing with behaviours, thoughts, or feelings that can vary in frequency or intensity (e.g. the Likert Scale\cite{likert1932technique}) and using open ended questions in cases where further insight is desired.

%Based on these guidelines two qualitative questionnaires have been designed, one aiming to gather insight into the sample demographic, the other aiming to gain insight into the participants' impressions about the realism of the experiment setup, the level of comfort, or discomfort and any suggestions or comments they might have. 

%\subsection{Participant Demographic}
%
%\subsection{Participant Feedback}

\section{Experiment Procedure}
\label{sec:meth-experiment-structure}
In this section, we describe in detail the experiment procedure. Participants are gathered through various methods, ranging from word-of-mouth to mailing lists, where each participant would reserve an experiment time slot. Participants are split randomly into two groups. The first group is referred to as the \emph{feedback group}, the second as the \emph{control group}. 

%In order to evaluate the effectiveness of the system a user study took place. Participants were split randomly into two groups. One group will be referred as the feedback group, the other will be referred to as the base group. The experiments structure was subdivided into smaller systemic tasks.

\begin{description}
	\item[Introduction] Each participant is introduced to the setup and given an overview of the experiment procedure. 

	\item[Demographic Questionnaire] The experiment starts by the participant responding to a brief questionnaire, aimed to gather more insight about the general participant demographics.
	
	\item[Rig Configuration] 
	The user climbs inside the rig; the seating position is adjusted to accommodate the user, making sure he or she is sitting comfortably, and all controls can be reached with ease. Participants are given a second, more in-depth overview of the components of the rig and how they operate. This explanation covers the steering wheel, the pedal layout and function and the H-shifter.
	
	\item[Practice (10 minutes)] The participant is given ten minutes to get used to the rig setup, track and car. This session is aimed at gauging the skill level of the user while he or she gets acquainted with the simulator.
	
	\item[Break (5 minutes)] A short break (5 minutes maximum) is given to each participant.
	
	\item[$1^{st}$ Session (10 minutes)] In this session, the participant drives the racing car around the track for ten minutes. Participants in the feedback group have the feedback system turned on, while for the control group, this is turned off.
	
	\item[Break (5 minutes)] A short break (5 minutes maximum) is given to each participant.

	\item[$2^{nd}$ Session (10 minutes)] A second ten minute session is held; participants in the feedback group have the feedback system turned on, while for the control group, this is turned off.
	
	\item[Break (5 minutes)] A short break (5 minutes maximum) is given to each participant.

	\item[$3^{rd}$ Session (10 minutes)] A third ten minute session is held; participants in the feedback group have the feedback system turned on, while for the control group, this is turned off.
		
	\item[Break (5 minutes)] A short break (5 minutes maximum) is given to each participant.
	
	\item[$4^{th}$ Session (5 minutes)] A final five minute driving session; the feedback system is turned off for participants in both groups. This session was included to help identifying conclusive results about the feedback system and its effects on the participants. 
	
%	\item[Five minuties session] A final five minuties of driving are allocated yet again, this time with the feedback system turned of for both groups. This was designed to possibly identify any conclusive results. Such session could show the possibility of the feedback group performing worst after having the aid of the feedback system removed or both groups ending up performing the same after the sessions, which would suggest the feedback participant didn't manage to get any cognitive advantage.
	
	\item[Feedback Questionnaire] The experiment concludes by the participant responding to a questionnaire about experiment structure, apparatus quality, performance perception and free-form participant feedback.
	
%	\item[Particpant's feedback questionare] The final stage of the experiment requires participants to fill in the a questionnaire  in which they are asked to give their feedback on the experiment structure, hardware used and any further comments they would like to add.
	
\end{description}

\section{Data Collection and Sampling}
\label{sec:meth-data-gathering}
The data collected during the experiments consists of two questionnaires per participant and four batches of telemetry data per participant. Each batch represents the telemetry data collected in a particular driving session. Google Forms was used to host the questionnaires due to its ease of use and the ability to export data and generate descriptive statistics on that data, such as frequency counts, etc.  

%At the end of the experiments the data collected includes two questioners from each participant and four batches of telemetry data, one for each participant. Questioners are filled online using Google Forms as it provides the ability to export the data and also automatic generation of descriptive statistic. The data collected from the questionnaires and telemetry data is to be loaded into a data base management system from which the data can be queried using specialised data querying constructs. By having a querying language, it provides the flexibility of extracting data which is relevant for the data analysis at hand.

\section{Data Analysis}
\label{sec:meth-data-analysis}
In order to evaluate the performance of each of the two groups (control and feedback) and perform meaningful statistical comparisons between them, the collected data were subjected to Independent T-Test\cite{student1908probable} and the Mann-Whitney U Test\cite{mann1947test}. These are appropriate tests for independent groups \cite{de2015statsref}. In our case, the groups are indeed independent because a participant from the control group may not make part of the feedback group and vice versa. The null hypothesis is that there is no significant difference across the groups, while the alternative hypothesis is that there is significant difference across the groups. The tests are described below:

%In order to accept or reject the null hypothesis statistical test are to be carried out. Most important is the ability to compare the performance of the two groups across sessions. The lap time is used as the test variable as this gives a good indication for the average performance achieved during a lap. Furthermore which comparison test to use depends on the sample size, the distribution of data and the types of group. In this case the groups are independent from each other, as a participant may not be part of the base group and also part of the feedback group. As pointed out by De Smith in his book "Statistical Analysis Handbook-a web-based statistics" the tests of interest to this project are the Independent samples t-test\cite{de2015statsref} and the Mann Whitney U test\cite{de2015statsref} as these test for difference between independent groups. Both tests share the same hypothesis listed below.

\begin{description}
	\item[Independent T Test] This test determines whether there is a statistically significant difference between the means in two unrelated groups. This test assumes the data is normally-distributed; this is verified by testing each group for normality using the Shapiro-Wilk\cite{shapiro1965analysis} test.
	
%	assumes the data is normally as such the data must be checked for normal distribution. This can be carried out by using the Shapiro-Wilk\cite{de2015statsref} test for normality on both groups.
%	
	\item[Mann-Whitney U test] This test does not require the data to be normally distributed; it does, however, assume that the data in the groups share the same distribution. This is verified by using the Kolmogorov-Smirnov \cite{kolmogorov1933sulla} test.
	
%	Does not require the data to be normally distributed however, it assumes the groups' data shares the same distribution. The groups are checked for equal distribution using the Independent Samples Kolmogorow-Smimov \cite{xxx} test.
\end{description}
	
%	\item[Null hypotesis]there is no significant difference across the groups
%	\item[Alternative hypotesis]there is significant difference across the groups
%\end{description}

%As the data distribution is not known before hand, distribution tests will be carried out after the data is collected and depending on the result, the adequate test will be used.

\section{Summary}
This chapter has given an in-depth overview of the methodology employed in this work. The methodology provides a detailed exposition on the experimental methods used, the hardware and software selection for the experiment and the data gathering and analysis phases for establishing the validity of our hypothesis.

\newpage
\section{Implementation}

\subsection{Supporting tools}

\subsection{Track Splicer}
This project is designed to work with any race track given specific track meta data is provided to aid the feedback system processing. The meta data is split into two files, ‘raceline.csv’ and ‘sections.csv’. ‘raceline.csv’ is a coma separated file containing a list of records denoting coordinates on track which make up the race. For this particular study the race line files have been generated from an .ai file which is supplied with Assetto Corsa. Each track has an associated 'ideal\_line.ai' file associated with it. The ai file contains raw bytes, which through manual investigation of the file in hex view, it was noted the file is made of a header part of 36 bytes, followed by a sequence of repeating records of 20 bytes each. These records contain four floats and one 32bit integer, storing the data which is required in the ‘raceline.csv’ file. The records in the ai file are read via a custom developed command line tool and translated into the csv format required by the feedback system. 	

\begin{center}
    \begin{tabular}{ | l | l | p{5cm} |}
    \hline
    Field Name & Description \\ \hline
    ID & Ordered unique ID \\ \hline
    Distance & The distance from the start of the lap \\ \hline
    X & X coordinates \\ \hline
    Y & Y coordinates \\ \hline
    Z & Z coordinates \\ \hline
    \end{tabular}
\end{center}

Moving on to the ‘sections.csv’, this file is also a coma separated file containing a sequence of records. These records denote corners and straights which make up the track, and will be used to compute any feedback which is specific to straights or corners. In order to generate this file a tool has been developed which loads the ‘raceline.csv’ and computes the rate of change from one data point to the next. Depending on the rate of change the points are classified as either part of a straight section or as a corner section. This is done by taking three points, p1, p2 and p3 from which two vectors are generated v1 and v2. V1 is the vector from p1 to p2, and v2 is the vector from p2 to p3. Then, v1 and v2 are normalised and the dot product computed which give out the rate of change in radians. The pseudo code for this is shown below.\\

for (i = 0; i < racelinePoints.Count - 2; i++)\\
\{\\
Vector2 v1 = racelinePoints[i].GetVectorToPoint(racelinePoints[i+1]);\\
Vector2 v2 = racelinePoints[i+1].GetVectorToPoint(racelinePoints[i+2]);\\
V1 = Vector2.Normalize(v1);\\
V2 = Vector2.Normalize(v2);\\
float dotProduct = Vector2.Dot(v1, v2);\\
double difference = Math.Acos(dotProduct);\\
\}\\

The corner mid-point can be defined as the highest section of the arch. In order to find this, it is simply a matter of finding the highest possible vector dot product from the section starting point, to the end of the section. The pseudo code is shown below.\\

Point p = endPoint - startPoint;\\
Vector2 n = new Vector2(-p.Y, p.X);\\
Int idOfMax = -1;\\
float max = -1;\\
for (i = trackSection.StartPoint; i <= trackSection.EndPoint; i++)\\
\{\\
p = \_RacingLine[i] - startPoint;\\
float result = Vector2.Dot(new Vector2(point.X, point.Y), n);\\
result = Math.Abs(result);\\
if (result > max)\\
\\{\\
max = result;
idOfMax = i;
\\}\\
\}

\ref{fig:TrackSplicerTool} shows the tool in action which also provides a visual representation of the race line. Corner sections are shows are show are red dots, with green dots used to highlight a corner’s mid-point and straights are shown in blue.

\begin{figure}[!htb]
	\centering
	\includegraphics[height=7cm]{images/tracksplicertool}
	\caption{Track splicer tool}
	\label{fig:TrackSplicerTool}
\end{figure}

\subsection{Spatial Querying}
As previously mentioned it is important for the feedback system to be able to carry out fast spatial querying operations. A query for the nearest race line data point based relative to the current position of the car is required to be carried out multiple times per second. Thanks to the implementation of a quad tree, the search guaranteed to take place in O(logn), while insertion is done O(nlogn) however, this is not too relevant as all insertion are carried out before the feedback system starts its computations. This structure allows the feedback to quickly calculate in which section of the track the car is located, the nearest race line data point and how far from the race line the car is.

\begin{figure}[!htb]
	\centering
	\includegraphics[height=7cm]{images/QuadTree}
	\caption{Visual representation for part of the quad tree}
	\label{fig:	QuadTree}
\end{figure}

VR Compatibility issues
VR not being able to render on screen native assetto corsa apps
Having to go with audio

Debugging and Testing
	Debug Info
	Unit Tests

\newpage
\section{Evaluation}

• One has to make sure and explain why all tests used to evaluate the system are relevant, using evidence from the literature about similar systems, and justifying any deviations from standard approaches;
• Demonstration that system works as intended (or not, as the case may be);
• Include comprehensible summaries of the results of all critical tests that have been made;
• If the student has not had time to carry out fully rigorous tests (in some cases, the student
may not have had time to produce a testable system) suggestions as to what tests would be and why they are relevant is important;
• The student must also critically evaluate the system in the light of these tests results, describing its strengths and weaknesses;
• Ideas for improving it can be carried over into the Future Work section;
• Comparison of practical with theoretical results and their interpretation. 

\section{Evaluation strategy}

\subsection{Experiment setup}

\subsection{User study}
Users are to be divided into two groups at random. One group of users will be asked to drive around the track without having any feedback provided by the system. This will evaluate how much a user can improve on their own. While the second group will also be asked to drive around, but this time the system will provide feedback on where and how the user can improve. A set of questions will be asked to the user once the test is complete. The questioner is meant to collect data on the users' racing experience prior to taking the test. Telemetry data will also be collected for both groups. Statistical analysis will be carried to determine if lap times do improve. 

\newpage
\chapter{Conclusion}
The Conclusions section should be a summary of the project and a restatement of its main results,
i.e. what has been learnt and what it has achieved. An effective set of conclusions should not
introduce new material. Instead it should draw out, summarise, combine and reiterate the main
points that have been made in the body of the dissertation and present opinions based on them. 

\section{Future Work}

Although the results presented here have demonstrated promising results and the system could be further developed in a number of ways:

\subsection{Suggest positive feedback}
At present the system only output feedback whenever a user does something wrong, an improvement to the system could explore the possibility of letting a user know whenever a previous mistake has been corrected,

\subsection{Visual and auditory hybrid feedback}
One can also look into implementing a hybrid feedback system, in which auditory feedback is aided by visual elements on screen. An example of such feature could be showing a user the slip ratio while braking. This could aid the user to better fine tune the braking as users would have a way to actually see how far off they are from the optimal braking slip ratio.

\subsection{\methodname control an AI car}
At present \methodname is tailored to aid users, however one could modify the output to control an AI car. This would work as a combination of neural net and fuzzy logic. In which the neural net is thought how to drive via the feedback system and the fuzzy logic is used to control the car. In this case the output would need to map to actual car controls inputs such as steering and pedals.

\subsection{Observer participant’s behavior while driving}
During the experiments it was noted that some lacked based skills such as keeping both hand on the wheel, not crossing hands while turning the wheel and resting the hand of the shifter. These events were outside of the scope of this system, and could not be monitored using telemetry data. However, one could use motion tracking camera to capture and report on such behaviors. 

\subsection{Teach users in the sim, have them test in real life}
The ultimate experiment for such a feedback system would be to have a set of participants who are trained in a virtual environment and then have them proof their learned skills in a real life environment. This would further validate the case for a racing serious game being used to teach users how to race.

There is clearly much work to be done in the area of, racing simulators and serious games. Perhaps the most direct extension of this work is by the means of using
a exploring any possibilities to further improve auditory feedback which is being provided at present 

\newpage
\bibliography{citeations}{}
\bibliographystyle{plain}

\newpage
\setcounter{page}{1}
\pagenumbering{roman} 
\begin{appendices}
	\chapter{}
%\label{chp:appendix}

\section{Further experiment results}

\begin{figure}[!htb]
	\centering
	\includegraphics[height=7cm]{charts/genderAge.pdf}
	\caption[Age and gender of experiments particpants]{Count Participants' gender and age }
	\label{fig:chart-genderage}
\end{figure}

\begin{figure}[!htb]
	\centering
	\includegraphics[height=7cm]{charts/licenseddriversexperience.pdf}
	\caption[Particpants Licensed drivers and driving experience]{Licensed drivers and experience}
	\label{fig:chart-licenseddriversexperience}
\end{figure}

\begin{figure}[!htb]
	\centering
	\includegraphics[height=5cm]{charts/playVideoGames.pdf}
	\caption[Do particpants play video games?]{Do particpants play video games?}
	\label{fig:chart-playVideoGames}
\end{figure}

\begin{figure}[!htb]
	\centering
	\includegraphics[height=5cm]{charts/gamesGenrePlayed.pdf}
	\caption[Racing games genre played by particpants]{Racing games genre played by particpants}
	\label{fig:chart-gamesGenrePlayed}
\end{figure}

\begin{figure}[!htb]
	\centering
	\includegraphics[height=5cm]{charts/usedARacingRig.pdf}
	\caption[Have particpants used a race rig?]{Have particpants used a race rig before the experiments?}
	\label{fig:chart-usedARacingRig}
\end{figure}

\begin{figure}[!htb]
	\centering
	\includegraphics[width=\textwidth]{charts/realistic.pdf}
	\caption[Participants rig realisim score]{How real do these componets feel?}
\label{fig:chart-realistic}
\end{figure}
%
%\begin{figure}[!htb]
%	\centering
%	\includegraphics[height=7cm]{charts/intrusivefeedback.pdf}
%	\caption{Was the feedback intrusive?}
%\label{fig:chart-intrusivefeedback}
%\end{figure}
%
%\section{Questionnaire}
%
%\includepdf[pages=-]{PreStudy.pdf}
%\includepdf[pages=-]{PostStudy.pdf}

\section{Transcript of the feedback audio files}

\begin{description}
	\item [Braking too hard] "Braking too hard"
	\item [Braking too light] "Braking too light"
	\item [Losing traction to the drive wheels by applying too much power] "Too aggressive, ease off the throttle"
	\item [Braking in corner] "Avoid braking while in a corner"
	\item [Incorrect race line during corner] "Turn in late into a corner, aiming for the inside apex"
	\item [Too aggressive during a corner] "Too aggressive during a corner, try easing off the throttle and using less steering input"
	\item [Too slow during a corner] "Try going faster during corner"
	\item [Changing gear too soon] "Changed gear too soon"
	\item [Changed gear too late] "Changed gear too later"
	\item [Taking too long to transition from one gear to another] "Took to long to change gear"
\end{description}
\end{appendices}

\end{document}