\documentclass{article}
\usepackage{cite}

\title{Telemetry-based Optimisation for User Training in Racing Simulators}
\date{Today}
\author{Francois Buhagiar}

\begin{document}

\maketitle
\pagenumbering{gobble}
\newpage
\pagenumbering{arabic}

\newpage
\section{Introduction}

\newpage
\section{Literature Review}

In the following section an explanation will be given for the ground work on which this thesis is based upon. The areas which will be covered include video games and serious games focusing on the differences between the two, which will be used to introduce the idea of using serious games as a training mechanism. Motorsport circuit car racing will also be discussed, describing what it involves from a formal point of view by defining the tasks which a racing driver is required to carry out in order to get a good lap time. 

\subsection{Video games and Serious Games}

Baranowski and colleagues defined games as “a physical or mental contest with a goal or objective, played according to a framework, or rule, that determines what a player can or cannot do inside a game world” the definition covers the setup of a game, while  "a physical or mental contest, played according to specific rules, with the goal of amusing or rewarding the participant"\cite{yuserious}.

Video games are built on top of these core values with the difference of having the game world confined into some sort of digital media. According to historians video games started with William Higinbotham who created a tennis game to be played on a television set\cite{stanton2015brief}. From the early days of video games, their main aim was always to provide some degree of entertainment. The entertainment value is achieved in various ways depending on gaming platform, game genre and the audience the video game is targeted to. According to Electronic Arts chief creative officer at the time, modern video games are simply made up of three fundamental components, story, art and software\cite{zyda2005visual}.

The definition of serious games has been redefined multiple times. The first formal definition appears to have been introduced by Abt in his book from 1970 which stated a serious game to be simulations and games to improve eduction\cite{abt1970}. Several years later, a white paper written by Sawyer in 2002 proposed an updated definition to be based on the idea of connecting a serious purpose to knowledge and technologies from the video game industry\cite{michael2005serious}. Moving on to nowadays definitions such as the ones from Chen and Michael in 2005\cite{michael2005serious} and from Zyda also in 2005\cite{zyda2005visual} seem to stem from Swayer's influence. The boundaries of serious games are debated, mostly due to the fact that serious game attract multiple domains making it hard to come up with a common boundary. However, the common denominator across all domains seems to be "Serious Game designers use people's interest in video games to capture their attention for a variety of purposes that go beyond pure entertainmnet"\cite{djaouti2011classifying}.

 	From the above one stands to reason the main contrast between video games and serious games involve the use of pedagogy activities that aim to educate or instruct knowledge or skill -\cite{zyda2005visual} in serious games. These activities are given preference over entertainment value, hence the amusement aspect which are custom to video games might not be found at all in a serious game\cite{zyda2005visual}. 

\subsection{Serious games as a training mechanism}

\subsection{Racing, getting near the optimal lap time}

\newpage
\section{Methodology}

\newpage
\bibliography{citeations}{}
\bibliographystyle{plain}

\end{document}