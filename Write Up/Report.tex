\documentclass{article}
\usepackage{cite}
\linespread{1.3}

\title{Telemetry-based Optimisation for User Training in Racing Simulators}
\author{Francois Buhagiar}
\date{2015}

\begin{document}

\pagenumbering{roman} 

\maketitle

\pagenumbering{arabic} 
\setcounter{page}{1}

\begin{abstract}
\end{abstract}

\newpage
\section{Introduction}

\newpage
\section{Literature Review}

In the following section an explanation will be given for the ground work on which this final year project is based upon. The areas which will be covered include video games and serious games focusing on the differences between the two, which will be used to introduce the idea of using serious games as a training mechanism. Motorsport circuit car racing will also be discussed, describing what it involves from a formal point of view by defining the tasks which a racing driver is required to carry out in order to get a good lap time. 

\subsection{Video games and Serious Games}

Baranowski and colleagues defined games as “a physical or mental contest with a goal or objective, played according to a framework, or rule, that determines what a player can or cannot do inside a game world” the definition covers the setup of a game, while  "a physical or mental contest, played according to specific rules, with the goal of amusing or rewarding the participant"\cite{yuserious}.

Video games are built on top of these core values with the difference of having the game world confined into some sort of digital media. According to historians video games started with William Higinbotham who created a tennis game to be played on a television set\cite{stanton2015brief}. From the early days of video games, their main aim was always to provide some degree of entertainment. The entertainment value is achieved in various ways depending on gaming platform, game genre and the audience the video game is targeted to. According to Electronic Arts chief creative officer at the time, modern video games are simply made up of three fundamental components, story, art and software\cite{zyda2005visual}.

The definition of serious games has been redefined multiple times. The first formal definition appears to have been introduced by Abt in his book from 1970 which stated a serious game to be simulations and games to improve eduction\cite{abt1970}. Several years later, a white paper written by Sawyer in 2002 proposed an updated definition to be based on the idea of connecting a serious purpose to knowledge and technologies from the video game industry\cite{michael2005serious}. Moving on to nowadays definitions such as the ones from Chen and Michael in 2005\cite{michael2005serious} and from Zyda also in 2005\cite{zyda2005visual} seem to stem from Swayer's influence. The boundaries of serious games are debated, mostly due to the fact that serious game attract multiple domains making it hard to come up with a common boundary. However, the common denominator across all domains seems to be "Serious Game designers use people's interest in video games to capture their attention for a variety of purposes that go beyond pure entertainmnet"\cite{djaouti2011classifying}.

From the above one stands to reason the main contrast between video games and serious games involve the use of pedagogy activities that aim to educate or instruct knowledge or skill -\cite{zyda2005visual} in serious games. These activities are given preference over entertainment value, hence the amusement aspect which are custom to video games might not be found at all in a serious game\cite{zyda2005visual}. 

\subsection{Consumer sim racing games as a serious game}

Sim racing games such as Asseto Corsa and Project Cars which are consumer available of the shelf, provide a sim racing experience within the average cost of other consumer games. The aim of sim racing games is to replicate real life cars, race car dynamics and track locations with the aim of providing entertainment and amusement to the player. The challenge aspect is achieved by paring the user against other AI players, multiplayer online races played against other human players, or sometimes against ones self. These points make racing games fit the previous definition of what a video game is however, fail to meet the requirements of a serious game, they miss the pedagogy activities. Most of the modern sim racing games do aid the player to improve by means of implementing aids. Such aids might include showing the racing line to which the player is expected to drive on, while also showing the braking and acceleration points. Other aids include anti lock brakes, traction control and stability control. The problem with their implementation is, the fact of them being implemented in a passive way. With the exception of the racing line, the player is not told when and what is being done wrong. The result is, users having to figure their own mistakes out by means of practicing without any guidance or feedback from with the game. This final year project aims to implement a module which is plugged into an off the shelf racing simulator which. This module trains users by letting them know what is being done wrong, when it's being done wrong and most importantly how to avoid making the same mistake.

[Show image of racing line red / green ?]
[In appendix, we might need to explain what ABS, TCS AND STC are]

\subsection{Racing, getting near the optimal lap time}
Talk about the racing aspect

\subsection{User study}
Maybe write something about how the approach the user study? Maybe find a Literature on how to ask user for their feedback which will be used for the survey after using the system.

\newpage
\section{Methodology}

\newpage
\bibliography{citeations}{}
\bibliographystyle{plain}

\end{document}