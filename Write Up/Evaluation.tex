\section{Evaluation}

• One has to make sure and explain why all tests used to evaluate the system are relevant, using evidence from the literature about similar systems, and justifying any deviations from standard approaches;
• Demonstration that system works as intended (or not, as the case may be);
• Include comprehensible summaries of the results of all critical tests that have been made;
• If the student has not had time to carry out fully rigorous tests (in some cases, the student
may not have had time to produce a testable system) suggestions as to what tests would be and why they are relevant is important;
• The student must also critically evaluate the system in the light of these tests results, describing its strengths and weaknesses;
• Ideas for improving it can be carried over into the Future Work section;
• Comparison of practical with theoretical results and their interpretation. 

\section{Evaluation strategy}

\subsection{Experiment setup}

\subsection{User study}
Users are to be divided into two groups at random. One group of users will be asked to drive around the track without having any feedback provided by the system. This will evaluate how much a user can improve on their own. While the second group will also be asked to drive around, but this time the system will provide feedback on where and how the user can improve. A set of questions will be asked to the user once the test is complete. The questioner is meant to collect data on the users' racing experience prior to taking the test. Telemetry data will also be collected for both groups. Statistical analysis will be carried to determine if lap times do improve. 