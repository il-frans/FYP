\chapter{Conclusion}
The Conclusions section should be a summary of the project and a restatement of its main results,
i.e. what has been learnt and what it has achieved. An effective set of conclusions should not
introduce new material. Instead it should draw out, summarise, combine and reiterate the main
points that have been made in the body of the dissertation and present opinions based on them. 

\section{Future Work}

Although the results presented here have demonstrated promising results and the system could be further developed in a number of ways:

\subsection{Suggest positive feedback}
At present the system only output feedback whenever a user does something wrong, an improvement to the system could explore the possibility of letting a user know whenever a previous mistake has been corrected,

\subsection{Visual and auditory hybrid feedback}
One can also look into implementing a hybrid feedback system, in which auditory feedback is aided by visual elements on screen. An example of such feature could be showing a user the slip ratio while braking. This could aid the user to better fine tune the braking as users would have a way to actually see how far off they are from the optimal braking slip ratio.

\subsection{\methodname control an AI car}
At present \methodname is tailored to aid users, however one could modify the output to control an AI car. This would work as a combination of neural net and fuzzy logic. In which the neural net is thought how to drive via the feedback system and the fuzzy logic is used to control the car. In this case the output would need to map to actual car controls inputs such as steering and pedals.

\subsection{Observer participant’s behavior while driving}
During the experiments it was noted that some lacked based skills such as keeping both hand on the wheel, not crossing hands while turning the wheel and resting the hand of the shifter. These events were outside of the scope of this system, and could not be monitored using telemetry data. However, one could use motion tracking camera to capture and report on such behaviors. 

\subsection{Teach users in the sim, have them test in real life}
The ultimate experiment for such a feedback system would be to have a set of participants who are trained in a virtual environment and then have them proof their learned skills in a real life environment. This would further validate the case for a racing serious game being used to teach users how to race.

There is clearly much work to be done in the area of, racing simulators and serious games. Perhaps the most direct extension of this work is by the means of using
a exploring any possibilities to further improve auditory feedback which is being provided at present 
