\chapter{Conclusion}
\label{chp:conclusion}
The skills required to become a good motorsport driver are generally learned through practice, and suggestions provided by more experienced drivers. Our hypothesis for this work is that an automated telemetry-based feedback system can be used to simulate this process and thus, to possibly help novice drivers assimilate this knowledge at a higher rate. For this purpose, \methodname was developed, which using a static expert system, presents auditory suggestions to drivers underlining the driving mistakes they are currently making. An experiment (see Chapter \ref{chp:methodology}) was set up to verify this hypothesis with 27 participants taking part. From the data gathered in this experiment, it appears that the real-time feedback was effective to some extent; however, the participants seemed to be having difficulty retaining and applying the suggestions once it was switched off. This lack of cognitive retention may be due to a number of reasons, which could be investigated in future work.

\section{Future Work}
Following on the results obtained, a number of research avenues which can provide additional insight into the process of learning motorsports skills have been identified. At present, \methodname presents feedback only when the driver is performing something wrong. It would be interesting to determine whether a system which also presents positive feedback, for instance letting the driver know that a previously reported mistake was actually corrected, would lead to any improvements. Furthermore, it is also possible to experiment with a variety of feedback methods, both visual and auditory, on the assumption that different feedback presentation media can lead to changes in the learning rates. Combining both auditory and visual feedback could help in increasing the clarity of the feedback provided. For instance, by visualising the slip ratio while braking, drivers would be able to see how far off they are from the optimal braking slip ratio. The feedback mechanism of \methodname is entirely based on telemetry data provided by the racing simulation software. However, a number of mistakes cannot be directly extracted from this data. For instance, during the experiment it was noted that some of the participants lacked some basic driving skills such as keeping both hands on the steering, not crossing hands while steering and not resting hands on the shifter. This additional information could be collected through the use of a motion tracking camera which directly feeds \methodname.   

\methodname is currently intended to help drivers improve their motorsport driving skills. A future direction could also look into the possibility of using \methodname to automate the driving process of a racing car. One possibility is that of using neural nets and fuzzy logic controllers, with both models learning how to drive via the feedback provided by \methodname. 

%The Conclusions section should be a summary of the project and a restatement of its main results,
%i.e. what has been learnt and what it has achieved. An effective set of conclusions should not
%introduce new material. Instead it should draw out, summarise, combine and reiterate the main
%points that have been made in the body of the dissertation and present opinions based on them. 
%
%\section{Future Work}
%
%Although the results presented here have demonstrated promising results and the system could be further developed in a number of ways:
%
%\subsection{Suggest positive feedback}
%At present the system only output feedback whenever a user does something wrong, an improvement to the system could explore the possibility of letting a user know whenever a previous mistake has been corrected,
%
%\subsection{Visual and auditory hybrid feedback}
%One can also look into implementing a hybrid feedback system, in which auditory feedback is aided by visual elements on screen. An example of such feature could be showing a user the slip ratio while braking. This could aid the user to better fine tune the braking as users would have a way to actually see how far off they are from the optimal braking slip ratio.
%
%\subsection{\methodname control an AI car}
%At present \methodname is tailored to aid users, however one could modify the output to control an AI car. This would work as a combination of neural net and fuzzy logic. In which the neural net is thought how to drive via the feedback system and the fuzzy logic is used to control the car. In this case the output would need to map to actual car controls inputs such as steering and pedals.
%
%\subsection{Observe participant’s behavior while driving}
%During the experiments it was noted that some lacked based skills such as keeping both hand on the wheel, not crossing hands while turning the wheel and resting the hand of the shifter. These events were outside of the scope of this system, and could not be monitored using telemetry data. However, one could use motion tracking camera to capture and report on such behaviors. 
%
%\subsection{Teach users in the sim, have them test in real life}
%The ultimate experiment for such a feedback system would be to have a set of participants who are trained in a virtual environment and then have them proof their learned skills in a real life environment. This would further validate the case for a racing serious game being used to teach users how to race.
%
%There is clearly much work to be done in the area of, racing simulators and serious games. Perhaps the most direct extension of this work is by the means of using
%a exploring any possibilities to further improve auditory feedback which is being provided at present 
