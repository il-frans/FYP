\documentclass{article}

\usepackage{color}
\usepackage{cite}
\usepackage{float}
\usepackage{graphicx}

\linespread{1.3}

\title{Telemetry-based Optimisation for User Training in Racing Simulators \\ Progress Report}

\author{Francois Buhagiar}
\date{2015}

\begin{document}

\pagenumbering{roman} 

\maketitle

\pagenumbering{arabic} 
\setcounter{page}{1}

\newpage
\begin{abstract}
\end{abstract}

\newpage
\section{Introduction}
	Race driving, it's training. Some games are being used in eductaion purposes this are called Serous games and how they they are branching into edu and more into racing domain. Our workd will .. Mention which circuit exist and that we will focus on tarmac.
	Set the base for the motivation

	\emph{To add}

\section{Motivation}
The training process for race drivers has stabilised during the last decade [cite], with rote learning playing a very important part. Starting at an early age, a driver would compete in lower leagues, such as go karting, and undergo training that is mostly founded on trial and error. A mentor, or coach, would correct obvious mistakes and suggest ways for improvement based on experiential knowledge and related literature. The extensive hours of practice serve to hone the skills of a driver and help in the acquisition of the same experiential knowledge of the mentor. Such learning methodology is very resource consuming in that it requires both time and money (justify why); often it is geographically-constrained as well, where no suitable training track is available in the locality of the driver. Although simulators, such as those employed by professional racing teams, have helped mitigating traveling and car setup times, they are inadequate for use in more amateurish environments due to cost and logistical problems: setting up such a simulator requires adequate space seldom available to everyone. Democratising the learning process such that proper car control and racing techniques can be mastered by a larger demographic an important motivation behind this work.

\emph Might be moved to problem defintion {Remove: This project will look into creating an ecosystem in which the software and hardware used as a backbone will be commercially available off the shelf. Reasoning being, since a racing simulator game is to be used one with simulated hardware once does not need to spend money on a race car and equipment. By keeping costs down learning proper racing techniques and car control will be more accessible to people which will hopefully end up developing the same sixth sense as other race drivers do.} \textbf{COMMANDMENT: 1. Never use sixth sense. 2. Never use ecosystem.}

\section{Why the problem is non-trivial}
The problem at hand is best described as an optimisation problem. Telemetry data provided by the car instrumentation system can be analysed to help identify driving patterns, specifically car-handling mistakes. The identification of these behaviours, which traditionally employs pattern recognition techniques, represents a challenge in itself. Behaviour recognition is key to providing corrective measures in order to improve the driving performance of a given user. In particular, it is the starting point in building a model which maps telemetry data to corrective measures for presentation to the user in real-time and deferred fashion, where even the visualisation of feedback is critical to the success of such a system. 

\textbf{Not needed: The driver could be making multiple mistakes at a given instance but it might be overwhelming showing all the data. For this reason, the system needs to identify which mistake is causing the driver to lose the most time while also taking into account how the difficult it is to master the skill which would eradicate the driver error. (I probably should go into more detail, correct?)}

\section{Background Work}

\subsection{Circuit motorsport racing, getting near the optimal lap time (Change heading)}

A sports race is defined as a situation in which individuals or groups compete to be first to achieve a particular objective \cite{mycitationsuckscock}. In the case of circuit motorsport races, in which motorised vehicles go round a course. Each racing discipline or series has its own rules. However, at the core, all disciplines participants aim to complete a full lap of the circuit in the least amount of time. Some disciplines focus on achieving one fast lap, such as time trials, while others focus on achieving the least amount of time across a fixed amount of laps, such as FIA's Formula 1 series. This dissertation will focus on confined car racing taking place on smooth asphalt surfaces in purpose built race tracks. 

\textbf{COMMANDMENT: 3. Do not to use thesis - use dissertation, work or project instead.}

A race driver has to figure out how to go round a piece of asphalt the minimum amount of time \cite{GoingFaster}. In order to do so, he or she needs to develop techniques for more advanced vehicle control. One such technique is that of mastering the \emph{race line}, which is considered the the fundamental skill a race driver must understand and master before moving on to anything else \cite{GoingFaster}. 

\textbf{COMMENT: EXPAND ON BOTH It is not clear whether the tricky part is figuring out the racing line in the corners, or knowing the racing line and following it in the corners.}

The best path through a corner is the one which takes the least time while keeping the higher average speed \cite{beckman1991physics}. The trickiest part of the racing line to master is that of a corner.

\subsection{Video games and Serious Games}

\textbf{Commandment: Do not put verbatim quotations in inverted commas - always paraphrase. Even if copied verbatim, do not enclose in quotes - most important thing is the citation following the text.}

Baranowski et al \cite{yuserious} define games as a physical or mental contest with a goal or objective, played according to a framework, or rule, that determines what a player can or cannot do inside a game world. The definition covers the setup of a game, while a physical or mental contest, played according to specific rules, with the goal of amusing or rewarding the participant the reward aspect of games.

Video games are built on top of these core values with the addition of having the game world confined to some sort of digital medium. The first video game was created by William Higinbotham; it was a tennis game to be played on a television set\cite{stanton2015brief}. From the early days of video games, their main aim was always to provide some degree of entertainment. The entertainment value is achieved in various ways depending on the gaming platform, game genre and the target audience. Modern video games are simply made up of three fundamental components: story, art and software \cite{zyda2005visual}.

\textbf {Clean up the following paragraph:}

Serious games, as a term, has been redefined multiple times. The first formal definition stated a serious game to be simulations and games to improve eduction \cite{abt1970}. Several years later an updated definition on the idea of connecting a serious purpose to knowledge and technologies from the video game industry\cite{michael2005serious}. Modern definitions seem to stem from the original definition\cite{michael2005serious}\cite{zyda2005visual}. The boundaries of serious games are debated, mostly due to the fact that serious games attract multiple domains making it hard to come up with a common boundary. However, the common denominator across all domains seems to be "Serious Game designers use people's interest in video games to capture their attention for a variety of purposes that go beyond pure entertainment"\cite{djaouti2011classifying}.

The main contrast between video games and serious games is the use of pedagogic activities which aim to educate or instruct knowledge or skill \cite{zyda2005visual} in serious games as opposed to the pure leisurely aspects of the video game. Pedagogy is given preference over the amusement value which in some cases might not be found in serious games \cite{zyda2005visual}.

\subsection{Simulation Racing as a Serious Game}

Simulation racing games (sim racing) such as Asseto Corsa \cite{aqqalla1} and Project CARS \cite{aqqalla2}, which are off-the-shelf products, provide a sim racing experience within budget for the average video game consumer. The aim such games is to replicate real life cars, race car dynamics and track locations to amuse and entertain the player. The challenge aspect is achieved by pitting the user against other computer drivers known as AI players, or in multiplayer online races, which are played against other human players. In some cases, a user can compete against oneself by taking on a ghost - a recording of the player's best lap for a particular track. 

\textbf{To clean:}

These points make racing games fit the previous definition of what a video game is however, fail to meet the requirements of a serious game, they miss the pedagogy activities. Most of the modern sim racing games do aid the player to improve by means of implementing aids. Such aids might include showing the racing line while also highlighting the braking and acceleration points. Other aids include anti lock brakes, traction control and stability control. The problem with their implementation is, the fact of them being implemented in a passive way. With the exception of the racing line, the player is not told when and what is being done wrong. This results in users having to figure out their own mistakes by means of practicing without any guidance or feedback from with the game. This final year project aims to implement a module which is plugged into an off the shelf racing simulator which. This module trains users by letting them know what is being done wrong, when it's being done wrong and most importantly how to avoid making the same mistake.

\textbf{Insert other driving serious games from literature (sort of related work)}

\section{Aims and objectives}

\textbf{Flesh the following more thoroughly. First sentence is the aim, others are objectives. Put objectives in point form and elaborate on each point individually. Make in point for } 

Train users in handling a racing car on a purpose built-tarmac circuit. This is to be achieved by looking into telemetry data from each individual user data which will result into a personalised feedback system. The feedback system’s aim is to let the user know which areas can be improved which should result into lower lap times. The main areas which the feedback system will focus on, are keeping the race line, braking and car control. At the end of each lap a scoring function will be applied to the lap telemetry data which will aid in highlighting if the feedback provided is helping the user or not.

\section{Methods and technologies used or planned}

\textbf{This needs a thorough revision. It does little to help whoever is reading the document to understand anything about the project. To me it seems more like a todo list you wrote for yourself, crumpled into paragraph form.}

\textbf{Remove: Methods and technologies have yet to be researched and confirmed. Expand on this and explain, include system overview diagram}
However, the plan is to use a UDP Client / Server communication to read the telemetry data. Quad trees to represent and query spatial data such as the racing line and the line taking by a user. Multiple interrelated variables need to be analysed in order to determine which area the user should focus on to improve. Fuzzy logic is planned to be used to determine what the user can improve in terms of racing. More importantly fuzzy logic will output the degree of gain the user will get from each minor improvement from which the system can choose which correction the user should be suggested. Data visualisation will also be looked into as the telemetry data in it’s raw form is not easy to understand at a glance. While the system needs to present the mistakes and corrections to the user in a away which can be easily understood. 

\section{Evaluation strategy}
Specify that a user study is going to be carried out to evaluate specific aspects of the system (possibly the benefits, etc)

\emph{This is the experiment setup:}
A simulation rig is to be setup in order to provide a sense of realism to the users which will be participating in the tests. The rig will be made of a racing wheel which provides force feedback, a three pedal set and a racing seat. Virtual reality will also be integrated into the system as to provide better sense of immersion. A race track and car will be preselected. This selection will be made based on ease of track layout and ease of car handling characteristics. 

\emph{This is the user study:}
Users are to be divided into two groups at random. One group of users will be asked to drive around the track without having any feedback provided from the system. This will evaluate how much a user can improve on their own. While the second group will also be asked to drive around, but this time the system will provide feedback on where and how the user can improve. A set of questions will be asked to the user once the test is complete. The questioner is meant to collect data on the users' racing experience prior to taking the test. Telemetry data will also be collected for both groups. Statistical analysis will be carried to determine if lap times do improve. 
\emph{Not clear what this sentence means:}
Though, are questions will also be able to get answered such us, if the second group users did take note of the feedback which was given to them by the system.

The result’s aim is to accept or reject the hypothesis of whether or not there is a significant improvement in lap times after a user uses the feedback system to learn racing techniques. 

\textbf{Latch this last sentence onto the main premise/aim of the project}

\section{Deliverables} Focus more on user study.

A plugin system developed with C\# .Net which is to be run in parallel with an off the self-racing simulator which is not yet confirmed. The system is to be made of three main components. The input interface layer which is used to connect to the racing sim telemetry data and translate it into a format which can be worked with by subsequent modules. The analysing module which takes care of identifying racing patterns and determining which areas the user needs to focus and improve on. The last module is the feedback module which handles the presentation of the results from the analysing module.
A user study will be carried out in order to validate the system. The user study reports will be part of the deliverables as well.

\section{Work Plan}

development
setup of use study
week user study paralle with write up
1 month write up

\newpage
\bibliography{citeations}{}
\bibliographystyle{plain}

\end{document}