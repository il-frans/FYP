\documentclass{sig-alternate}

\usepackage{color}
\usepackage{cite}
\usepackage{float}
\usepackage{xspace}
\usepackage{amsfonts}
\usepackage{graphicx}
\usepackage{pgfgantt}
\usepackage{setspace}
\usepackage{gensymb}
\usepackage{pdfpages}
\usepackage{eurosym}
\usepackage{booktabs}
\usepackage{hyperref}
\usepackage[toc,page]{appendix}
\usepackage{multirow}

\hypersetup{
	colorlinks,
	citecolor=black,
	filecolor=black,
	linkcolor=black,
	urlcolor=black
}

\begin{document}

\title{Telemetry-based Optimisation for User Training in Racing Simulators}

\numberofauthors{1} 
\author{
	\alignauthor
	Fran\c{c}ois Buhagiar\\
    \email{francois.buhagiar.12@um.edu.mt}
}

\maketitle
\begin{abstract}

Motorsports requires training and dedication to master, supplemented by hours of rote learning and mentoring by experts. This study explores the question of whether a serious game is a powerful enough pedagogical tool to be gainfully employed in the training of race drivers. A system of heuristics is proposed for a novel telemetry-based feedback model for contextual real-time suggestions. The model has been integrated into a racing simulation game and a study of its performance is reported here. The study consists of 27 participants, partitioned into two groups, to provide a control for the experiment. Two questionnaires have been used to acquire demographic information about the participants and help control for factors such as experience. Quantitative results show that there is an improvement for the group using the feedback system, although this improvement dissipates when the feedback is disabled again for the experimental group. Analysis of the initial results are encouraging, with the model showing promise. Additionally, the lack of cognitive retention on behalf of the participants when feedback was disabled merits further investigation and future work.

\end{abstract}

\keywords{Serious Games, Motorsports, Simulation, Learning, Training}

\def \methodname {TeAR\xspace}
\def \methodnamefull {Telemetry Assisted Racing\xspace}

\section{Introduction}
The gamification of areas of activity such as marketing, problem solving and education \cite{michael2005serious} has validated the use of serious games beyond their initial military use in training strategic skills \cite{djaouti2011classifying}.  Serious games simulate real-world processes designed for the purpose of solving a problem, making their main purpose that of training or educating users. Their popularity has been steadily increasing, as has their adoption, with military \cite{djaouti2011classifying} and emergency service providers (e.g. firefighters \cite{michael2005serious}) employing them to train for specific scenarios that might be encountered on the respective jobs. Motorsports cover a broad range of activities and vehicles, and as with all major forms of sporting activities, require training and dedication, with a pedagogic aspect arising in rote learning and mentoring by experts. The arenas in which motorsport events take place are called circuits; there is a large selection of the latter, ranging from purposely-built race tracks to public roads to natural formations such as hills and quarries. There is also a diverse selection of vehicles that take part in motorsports, with the greatest demarcation existing between motorbikes and cars. The focus of this dissertation is that of unifying serious games and motorsport racing; specifically, it will try to show whether a serious game is a powerful enough pedagogical tool that can be used to tangibly improve the performance of race drivers. The scope of the project is limited to four-wheeled cars racing on purposely-built confined circuits with a smooth tarmac surface.

\subsection{Aims and Objectives}
The aim of this work is to demonstrate whether serious games can be used to teach a particular brand of motorsports. In order to establish whether this is possible or otherwise, the following objectives have been set out:

\begin{enumerate}
	\item Research and develop a model for assessing race driver performance
	\item Develop a system of heuristics for providing feedback from the aforementioned model
	\item Devise an experimental methodology for evaluating how effective the feedback model is (if at all) in training race drivers
	\item Design and develop the main software and support tools for integrating the feedback model into a racing simulation game
	\item Carry out a user study based on devised experimental methodology to gather data for analysis
	\item Perform statistical analysis on the gathered data to answer research question
\end{enumerate}


\section{Background} {
\label{sec:background}
%
%In this section fundamental concepts relevant to this paper are given a brief overview. The section start by explaining what motor sports is and what is expected off a race drive. The section continues by mentioning fundamental concepts such as the racing line and how to achieve the limit of the car. An explanation of telemetry data is given and why it is important. The section concludes by explaining the differences between video games and serious games. 

% \subsection{Motorsport Racing}
In circuit motorsport racing, motorised vehicles go round a course for a set number of times. There are varies racing disciplines or series, each one having its own specific rules. However, at the core, participants in all disciplines aim to complete a full lap of the circuit in the shortest time. This paper will focus on one such discipline, that of confined car racing, which takes place on smooth asphalt surfaces in purpose-built race tracks in which the aim is to perform the fastest lap possible.

\subsection{Car and Circuit Dynamics}
\emph{A race driver needs to figure out how to go round a piece of asphalt in the minimum amount of time} \cite{GoingFaster}. In order to do so, he or she needs to develop techniques for more advanced vehicle control. One such technique is that of mastering the racing line, which is considered the fundamental skill a race driver must understand and master before moving on to anything else \cite{GoingFaster}. The racing line is the best path through a circuit: if followed, it is the path that yields the shortest time at the highest average speed \cite{beckman1991physics}. The trickiest part in achieving this is to master the racing line of a corner. Corners are split in three sections. The first section is the breaking part, where the car needs to sufficiently decelerate in preparation for the  \emph{turn-in point}. Thus, the second partition of the racing line at a corner is the segment between the turn-in point and the apex point, which is the inside mid-point of the corner (see Figure \ref{fig:CornerRaceLine}). After the turn-in point, the driver aims for the apex point. The final section of the racing line in a corner lies from the apex point onwards, where the driver must gradually accelerate out of the corner, while still turning, aiming for the outside apex (see Figure \ref{fig:CornerRaceLine}.

\begin{figure}[!htb]
	\centering
	\includegraphics[width=0.45\textwidth]{images/cornerraceline}
	\caption{Racing line through a 90" right corner}
	\label{fig:CornerRaceLine}
\end{figure}

%\subsection{Limit Of The Car}
As the driver gets acquainted to the racing line, usually at sub-optimal speeds, he must find the limit of the car, which is the highest speed the car can be driven while still retaining some measure of control. Various studies have been carried out to define such a limit in terms of the physical properties of the car and its environment \cite{beckman1991physics}. The most important property is the level of grip the car can achieve and sustain on track. A number of factors contribute to the level of grip. Most notably, one very important factor is the tyres as they are the only contact the car makes with the track, and allow for braking, accelerating and turning forces to be transferred to the asphalt.

Each tyre has two properties which are of particular interest: the slip ratio and slip angle. The slip ratio refers to the level of slip occurring between tyre rotational velocity and the asphalt which occur during acceleration and deceleration The \emph{slip ratio} is expressed as a percentage: a slip percentage of 100\% means that the tyre is rotating but the road is stationary. In jargon, this is called \emph{wheel spin}. On the other hand, a percentage of -100\% indicates that the tyre is not rotating but the road beneath is moving. This can occur when braking too hard and is called \emph{locking the wheels}. Both events need to be avoid by the driver as they not only put excessive wear and tear, they also impact negatively the performance of the driver. For an optimal braking procedure, the slip ratio should be between 10\% to 15\% \cite{GoingFaster} and no slip should occur during acceleration.

The slip angle is the angle between the tyre's desired direction (perpendicular to the axis of rotation of the tyre) and the tyre's actual direction (the direction the car is moving in). Given both the actual direction of travel ($\mathbf{d}_t$) and the desired direction ($\mathbf{d}_d$) are known, the slip angle $s_a$ is calculated as follows:
\begin{equation}
s_a = \cos^{-1}(\hat{\mathbf{d}}_d \cdot \hat{\mathbf{d}_t}),
\end{equation}
\noindent where $\hat{\mathbf{d}_d} = \frac{\mathbf{d}_d}{|\mathbf{d}_d|}$ and $\hat{\mathbf{d}_t} = \frac{\mathbf{d}_t}{|\mathbf{d}_t|}$ are the normalised direction vectors for desired and travel directions respectively.

Whenever the slip angle is above $0\degree$ ($s_a > 0$) the tyre is said to be in an understeering situation. Understeer can be caused by active factors such as cornering speed, throttle application, braking, steering inputs and weight transfer. A tyre has an optimal slip angle at which grip is maximised during cornering. The optimal slip angle for a road tyre is about $5\degree$, whereas for a slick tyre, which is purposely constructed for racing, is about $8\degree-10\degree$ \cite{beckman1991physics}.

An oversteering situation may arise from lack of grip; while understeer is caused by a lack of grip in the front tyres, oversteer is cause by a lack of grip on the rear tyres. Oversteer is usually denoted by the rear of the vehicle becoming unstable resulting in its rotation such that the driver is facing towards the inside of the corner. Similarly to understeer, the active factors causing oversteer are also cornering speed, throttle application, braking, steering inputs and weight transfer. Oversteer is usually induced by braking during a corner or accelerating too hard in a rear wheel drive vehicle.

\subsubsection{Telemetry Data}
In motorsport, telemetry data contains measurements of vehicle dynamics from the engine and other components and is transmitted to receiving equipment for remote monitoring. These measurements can serve to monitor and reconstruct the vehicle state at a particular point in time. Telemetry data in motorsports usually accounts for measurements of speed, engine speed, component temperatures, slip angles, slip ratios, etc. Telemetry is widely regarded as the most important source of information by motorsports engineers; analysing this data can lead to a better understanding of the respective strengths and weaknesses of the car and the driver \cite{CarDataAnalysis}. In this work, we posit that through the real-time analysis of telemetry data, the pedagogical aspect of sim racing can be exploited to teach race driving to non experts.

\subsection{Racing Simulation Rigs}
The racing simulation rig (sim racing rig) is a piece of equipment designed to mimic the cockpit of a real-world car. The quality of a sim racing rig is dependent on its authenticity - how similar it is to a real-world car - and its build quality. These rigs come in various shapes, forms and sizes, from hangar-sized hydraulic-driven car chassis, that cost millions of Euro, to the more modest, built from off-the-shelf commodity hardware. Minimally, a rig should provide a steering wheel, seating and a display. More sophisticated rigs augment the user experience by employing gear shifters, and clutch, throttle and breaking pedals. The more advanced components are furnished with a force feedback mechanism, a form of haptic technology used to replicate the sense of touch by applying forces or vibrations, or motions to the user \cite{li2015can}.

\section{Related Work}
\label{sec:related-work}
Baranowski et al.~\cite{yuserious} define games as a physical or mental contest with a goal or objective, played according to a framework, or rule, that determines what a player can or cannot do inside a game world. Video games are built on top of these core values with the addition of having the game world confined to some sort of digital medium. From the early days of video games, their main aim was always to provide some degree of entertainment~\cite{stanton2015brief}. Modern video games are made up of three fundamental components: story, art and software \cite{zyda2005visual}.

Serious games leverage traditional video games and the respective technologies to improve education~\cite{abt1970, stanton2015brief}. Djaouti et al. hold that the ability of video games to capture the attention of users can be harnessed for a variety of purposes that go beyond entertainment. \cite{djaouti2011classifying}. Thus, the main differentiator between video games and serious games is the use of pedagogic activities that aim to educate or instruct the player~\cite{zyda2005visual, Harteveld2007}. In order to produce a valid pedagogical experience, aspects such as learning objectives, target groups and challenges need to be clearly identified during the design process of a serious game~\cite{moser2002methodology}. Egentfeld-Nielset et al. identify three important properties that can help in this process:
\begin{description}
\item[Experience] Games tend to provide learning-by-doing~\cite{egenfeldt2005beyond}.

\item[Exploration] The game should encourage the player to adopt an active and participative attitude. 

\item[Incremental] The learning process should occur incrementally to avoid taxing the player's abilities~\cite{moser2002methodology}.
\end{description}

Racing simulation games such as Asseto Corsa~\cite{assestoCorsa} and Project CARS~\cite{ProjectCars}, while providing the player with various driving aids such as the racing line, traction control and stability, do not actively engage in providing a pedagogic experience to the user; instead, any learning experience is passively transferred, since the player is never made aware of his mistakes. Vogel et al.~\cite{vogel2006computer} have provided a meta-analysis of computer games and simulations for learning; while Li et al.~\cite{li2015can} have specifically investigated how car driving in computer games can provide traffic education to users, the authors are not aware of any other work that studies serious games and motorsports training.

\section{Methodology}
\label{sec:Methodology}
This section introduces the methodology used for this experimental study, followed by an overview of the hardware and software used during the experiments, and an explanation of the experiment procedure. The section concludes with the statistical tests chosen for analysis.

\subsection{Experiment Design}
In the experiment, the same car and racetrack will be utilised for all participants. Bastow et al. \cite{bastow2004car} suggest that cars equipped with a front wheel drivetrain may be easier to handle, and motivated in part by these findings, the Fiat Abarth 500 was chosen. It is a relatively low-powered car and thus, easier to use by novice drivers. The Silverstone National race track has the desirable properties of being flat and smooth, without uneven surfaces or bumps which may result in loss of control in unexperienced drivers. Furthermore, the way the track is structured, with wide run-off areas located along the circuit where drivers are most likely to lose control of the car, allows the car to slow down before colliding with barriers or other stationary objects. Two feedback mechanisms have been considered for this experiment, \emph{visual}, through the use of a heads-up-display (HUD) superimposed on the simulation display, or \emph{auditory}, by means of descriptive speech projected through loud speakers. Leahy et al. \cite{leahy2003auditory} argue that auditory feedback is less intrusive than visual clues; based on these findings, it was decided that the system should provide feedback using auditory clues.

\subsection{Experiment Materials}
The setup of the experiment was divided into three material categories: \emph{simulation environment}, \emph{simulation hardware} and \emph{simulation software}, with each category subscribing to a number of desirable properties:

\begin{description}
	\item [Environment] The experiment should be carried out in an isolated, noise-free and well-lit room. Participants would be let in the room one at a time, to ensure the experiment is conducted without any distractions.
	\item [Hardware] The hardware components identified for this experiment are the (i) display output, (ii) audio output, (iii) steering wheel, (iv) gear shifter, (v) acceleration, brake and clutch pedals and (vi) seating frame.
	\item [Software] The software aspect of the experiment, which is primarily the racing simulator, should have a number of desirable properties. In particular, the simulator should provide (i) a realistic driving model, (ii) driving aid customisation, (iii) high-fidelity graphics, (iv) real-world tarmac circuits, (v) and telemetry data access through an interface that is intuitive to use.
\end{description}

\subsection{Experiment Procedure}
Participants are gathered through various methods, ranging from word-of-mouth to mailing lists, where each participant would reserve an experiment time slot. Participants are split randomly into two groups. The first group is referred to as the \emph{feedback group}, the second as the \emph{control group}.

\begin{description}
	\item[Introduction] Each participant is introduced to the setup and given an overview of the experiment procedure. 
	
	\item[Demographic Questionnaire] The experiment starts by having the participant responding to a brief questionnaire, aimed to gather more insight about the general participant demographics.
	
	\item[Rig Configuration] 
	The user climbs inside the rig; the seating position is adjusted to accommodate the user, making sure he or she is sitting comfortably. Participants are given a second, more in-depth overview of the components of the rig and how they operate.
	
	\item[Practice (10 minutes)] The participant is given ten minutes to get used to the rig setup, track and car.
	
	\item[Break (5 minutes)] A short break (5 minutes maximum) is given to each participant.
	
	\item[$1^{st}$ Session (10 minutes)] In this session, the participant drives the racing car around the track for ten minutes. Participants in the feedback group have the feedback system turned on, while for the control group, this is turned off.
	
	\item[Break (5 minutes)] A short break (5 minutes maximum) is given to each participant.
	
	\item[$2^{nd}$ Session (10 minutes)] A second ten minute session is held; participants in the feedback group have the feedback system turned on, while for the control group, this is turned off.
	
	\item[Break (5 minutes)] A short break (5 minutes maximum) is given to each participant.
	
	\item[$3^{rd}$ Session (5 minutes)] A final five minute driving session; the feedback system is turned off for participants in both groups.
	
	\item[Feedback Questionnaire] The experiment concludes by the participant responding to a questionnaire about experiment structure, apparatus quality, performance perception and free-form participant feedback.
	
\end{description}

\subsection{Data Collection, Sampling and Analysis}
The data collected during the experiments consists of two questionnaires per participant and four batches of telemetry data per participant. Each batch represents the telemetry data collected in a particular driving session. In order to evaluate the performance of each of the two groups (control and feedback) and perform meaningful statistical comparisons between them, the collected data were subjected to Independent T-Test\cite{student1908probable} and the Mann-Whitney U Test\cite{mann1947test}. These are appropriate tests for independent groups\cite{de2015statsref}. In our case, the groups are indeed independent because a participant from the control group may not make part of the feedback group and vice versa. The null hypothesis is that there is no significant difference across the groups, while the alternative hypothesis is that there is significant difference across the groups.

\section{Telemetry Assisted Racing}
\label{sec:DesignImplementation}

This section gives a brief overview of \emph{\methodnamefull (\methodname)}, the software artefact developed for this work. The system architecture of \methodname is shown in Figure \ref{fig:SystemArch}; \methodname is comprised of three main modules: (i) the Telemetry Input Interface, (ii) the Processing module and (ii) the Output Interface. 

	

%\subsection{Content Creation and Annotation}
%The telemetry data stream read-out from Assetto Corsa does not provide sufficient information about the environment for \methodname to make informed decisions about possible feedback. The feedback mechanism used in \methodname is highly dependent on the racing line, which in turn, is a function of the track geometry. The creation of the \emph{Track Splicer and Visualiser}, a supporting tool that extracts track information from the Assetto Corsa data files and provides additional annotations to this data was required by \methodname.
%
%\subsection{Persistence of Telemetry Data}
%Log files containing telemetry data from each user session are inserted into a database management system for offline analysis. An extraction transfer loading (ETL\cite{kimball2004data}) process was purposely built, which takes as input a log file and inserts processed data as records into a database. This database is then used to run SQL queries related to the evaluation of the experiment.

%\subsection{System Architecture}

\begin{figure}[!htb]
	\centering
	\includegraphics[width=0.45\textwidth]{diagrams/SystemArch}
	\caption{Overview of the system architecture components}
	\label{fig:SystemArch}
\end{figure}

The \textbf {Telemetry Input Interface} is responsible for handling the telemetry data stream coming from the racing simulator. The module also handles supporting files that place telemetry data into context, such as race track geometry and its respective annotations (e.g. racing line). Communication with the racing simulator is handled using UDP sockets. 
The telemetry data is then filtered and rearranged for further processing by the \textbf{Processing} module, which coordinates a number of sub-modules each consuming the same telemetry data in a different way (e.g. Logger, Car Handling Feedback). These sub-modules run on separate threads and communicate and synchronise with the main Processing module using shared memory. Any feedback notification raised by the sub-modules is captured by the processing module, which in turn forwards the message to the output module. The feedback processing sub-modules possess a common structure; each module is bound to a telemetry event. These events may be triggered by changes in one or more of the incoming telemetry data values. When an event is triggered, the respective sub-module will start the analysis process related to the respective telemetry information. Eventually, when the corresponding termination event (not necessarily tied to the same telemetry data as the activation event) is triggered, a decision is taken as to whether provide feedback or not, and if so, which feedback. The system operates using two feedback tiers; initially, at the first tier, only feedback related to the racing line is provided. As the user masters the racing line, the system will switch to the second tier and suggest more advanced feedback.

The final component in \methodname's pipeline is the \textbf{Output Interface}, through which feedback suggestions coming from the Processing module are presented to the user. This module connects to a repository of audio soundbites that map to the input feedback from the previous module. The audio files in the repository were generated using a free online text-to-speech tool\footnote{http://www.fromtexttospeech.com}. The module does not support concurrent playback of audio feedback by design, since this would overwhelm the user. Furthermore, since feedback is topical, buffering of feedback was intentionally omitted.

\section{Evaluation} 
\label{sec:Evaluation}
In this section findings from the user study are presented. The user study has been carried out in line with the methodology described in \S\ref{sec:Methodology}. A custom-built simulation racing rig was used for this experiment, with Logitech G25 input devices mounted on it (see Figure \ref{fig:eval-simRacingRig}). A 32" LCD HD Samsung display with 2x10W speakers was employed for visual and auditory output. The racing simulator used was Assetto Corsa by Kunos Simulazioni, running in full HD resolution at 60 frames per second (fps). The hardware platform running the simulator was an Intel Core i7-940, with 8GB of RAM and an Nvidia 660Ti video card with 2GB VRAM. The machine was running the retail version of the Windows 10 Professional operating system.

\begin{figure}[!htb]
	\centering
	\begin{minipage}{0.45\textwidth}
		\centering
		\includegraphics[width=\textwidth]{images/RacingRig}
	\end{minipage}\hfill
	\caption[Side view of the racing rig]{Side view of the racing rig}
	\label{fig:eval-simRacingRig}
\end{figure}

\subsection{Sample Demographic}
27 participants took part in the user study, mostly males in their early twenties. Out of 27 participants, 2 did not hold a driving license; 25 had a driving license, although most of them haven't been driving for more than a year. 22 participants claimed to play video games, from which 18 stated to have played racing video games. Out of these 18, a majority of 15 play mostly arcade sim racing games, while the remaining 3 regularly play simulation racing games. 7 out of 27 participants have previously used a racing rig. Random assignment was used to split the participants into the control and feedback/experimental group. The control group had 13 participants, while the feedback group had 14.

\subsection{Experiments Results}

\begin{figure}[!htb]
	\centering
	\includegraphics[width=0.45\textwidth]{charts/laptimes.png}
	\caption[Lap times vs session, clustered by group]{Blue : Control Group, Green : Feedback group \\ Lap times vs session, clustered by group}
	\label{fig:chart-laptimes}
\end{figure}

The dataset of the 1st session shown in Figure \ref{fig:chart-laptimes} was subjected to the Kolmogorov-Smirnov Test; the result had a p-value of 0.360, with a significance level of 0.05 the null hypothesis is accepted. The groups share the same distribution, hence the Mann-Whitney U Test can be used.

\begin{table*}
	\centering
	\begin{tabular}{|l|l|l|l|l|}
		\hline
		& \multicolumn{4}{c|}{LapTime}            \\ \cline{2-5} 
		& 1st Session & 2nd Session & 3rd Session & 4th Session \\ \hline
		Asymp. Sig. (2-tailed) & .057 & .054        & .029        & .539        \\ \hline
	\end{tabular}\\
	Grouping Variable : Group
	\caption[Mann-Whitney U Test for Experiments' Sessions]{Mann-Whitney U Test result for Experiments' Sessions}
	\label{table:Mann-Whitney-Sessions}
\end{table*}

The Mann-Whitney Test is used to determine differences in the groups. The results of the test are shown in Table \ref{table:Mann-Whitney-Sessions}. For the first session these show a p-value of 0.057, with a confidence interval of 0.95 resulting in no statistical difference between the groups (p-value $> 0.05$, thus accepting the alternative hypothesis).  The tests for the second and fourth session have a p-value of 0.054 and 0.539 respectively; both values are above the significance level of 0.05, resulting in the alternative hypothesis being accepted. The third session, however, has a p-value of 0.029, resulting in accepting the null hypothesis suggesting there is statistical difference in the groups' performance.

\subsection{Participant Feedback}
The participants reported a good experience overall. The rig setup was found to be realistic and easy to use. With respect to the difficulty of the race track, an overwhelming majority of the respondents reported having issues mastering the second and third corners (referred to as the \emph{s-bend}). When the feedback group was asked about \methodname, they reported the feedback and the output quality of the audio to be intelligible, accurate and helpful. They also said that the feedback hints were somewhat easy to apply. Lastly, when asked whether the feedback was intrusive and possibly distractive, 10 out of 14 respondents reported that it was neither intrusive nor distractive.

\subsection{Discussion}
The rig setup has been well received, with participants enjoying the driving and awarding it a high score in terms of realism. This suggests that is it possible to achieve a good level of realism using simple off-the-shelf hardware. 
Both groups show a noticeable improvement from one session to the next, except for the final session in which the shorter time slot allocated might have put additional pressure on the participants, hindering their performance.
For the second session, the tests shown no statistical differences in the performance of the two groups, albeit this changed for the third session. The lower average lap time for the feedback group suggests that after familiarising with the feedback system, the group started following its instructions more closely, thus improving their performance.

The available sample of participants is too small to test for correlation between gaming experience and lap times, The same argument can be made for a possible correlation between holding a driving licence and driving performance, since all but two participants are in possession of a driving licence.
}

\section{Conclusions} {
\label{sec:Conclusions}
The skills required to become a good motorsport driver are generally learned through practice, and suggestions provided by more experienced drivers. Our hypothesis for this work is that an automated telemetry-based feedback system can be used to simulate this process and thus, to possibly help novice drivers assimilate this knowledge at a higher rate. For this purpose, \methodname was developed, which using a static expert system, presents auditory suggestions to drivers underlining the driving mistakes they are currently making. An was set up to verify this hypothesis with 27 participants taking part. From the data gathered in this experiment, it appears that the real-time feedback was effective to some extent; however, the participants seemed to be having difficulty retaining and applying the suggestions once it was switched off. This lack of cognitive retention may be due to a number of reasons, which could be investigated in future work.

\subsection{Future Work}
At present, \methodname presents feedback only when the driver is performing something wrong. It would be interesting to determine whether a system which also presents positive feedback, for instance letting the driver know that a previously reported mistake was actually corrected, would lead to any improvements. Furthermore, it is also possible to experiment with a variety of feedback methods, both visual and auditory, on the assumption that different feedback presentation media can lead to changes in the learning rates. Combining both auditory and visual feedback could help in increasing the clarity of the feedback provided. The feedback mechanism of \methodname is entirely based on telemetry data provided by the racing simulation software. However, a number of mistakes cannot be directly extracted from this data. For instance, during the experiment it was noted that some of the participants lacked some basic driving skills such as keeping both hands on the steering, not crossing hands while steering and not resting hands on the shifter. This additional information could be collected through the use of a motion tracking camera which directly feeds \methodname.   

\methodname is currently intended to help drivers improve their motorsport driving skills. A future direction could also look into the possibility of using \methodname to automate the driving process of a racing car. One possibility is that of using neural nets and fuzzy logic controllers, with both models learning how to drive via the feedback provided by \methodname.

}

\bibliographystyle{abbrv}
\bibliography{sigproc}

\end{document}
